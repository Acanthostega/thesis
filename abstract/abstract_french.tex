\bartchapterimage{heic0901a.jpg}
\chapter*{Résumé}
\label{cha:resume}
\bartthumb{heic0901a.png}

Les galaxies reposent dans un large éventail d'environnements allant des
galaxies isolées, aux paires, aux groupes ou amas. Il est donc légitime de
penser que cet environnement peut influer sur les différentes propriétés des
galaxies comme la morphologie, la formation stellaire, la métallicité, etc. Des
études ont déjà tenté de quantifier les rôles de l'environnement global (lié à
la masse du halo de matière noire du groupe) et de l'environnement local (la
position de la galaxie dans le groupe). Elles ont montré que l'environnement
joue un rôle mineur dans leurs propriétés excepté pour les galaxies de faible
masse. Mais la quantification de l'environnement est difficile car les groupes
détectés dans l'espace des redshifts (seul accessible à l'observateur) sont
très allongés et ne facilitent donc pas la recherche de l'appartenance d'une
galaxie à un groupe donné. Si ces erreurs de quantification sont trop
importantes, les effets de l'environnement seront alors mal mesurés.

De plus, d'autres processus physiques sont à l'œuvre dans les groupes dont
l'importance n'est pas tout à fait comprise. Par exemple les fusions majeures
ou mineures des galaxies (riches ou pauvres en gaz, entre galaxies non
centrales, ou entre une centrale et une non centrale par "descente" de son
orbite après friction dynamique), les survols rapides qui arrachent du gaz aux
galaxies, le dépouillement du gaz interstellaire par la pression du gaz
intra-groupe ou intra-amas, ou de celui du réservoir de gaz qui forme les
disques des galaxies par des effets de marées. Bien que les modèles
semi-analytiques de formation des galaxies à partir de conditions initiales
d'un Univers $\Lambda${CDM} représentent assez bien les observations faîtes sur
les galaxies, il y a toujours des écarts qui peuvent être sûrement liés à un
manque de prise en compte des effets d'environnement dans ces modèles.

On vise donc avec cette thèse à avoir une compréhension détaillée du rôle de
l'environnement sur les propriétés des galaxies et finalement connaître le ou
les processus physiques qui ont une importance prépondérante dans la modulation
de ces propriétés avec l'environnement local et global. Pour cela, il est
nécessaire de réaliser une extraction optimale des groupes de galaxies depuis
l'espace des phases projeté.

Une étude et ré-implémentation de certains algorithmes de regroupement de
galaxies déjà existants a été réalisée pour déterminer leur efficacité et leurs
faiblesses dans la détection des groupes de galaxies.

Un catalogue de galaxies test (mock catalogue) a été réalisé pour appliquer nos
divers algorithmes de regroupement sur un échantillon de galaxies certes
fictif, mais avec des propriétés physiques semblables (fonction de luminosité,
profil de densité des galaxies dans les groupes, biais liés au décalage vers le
rouge comme indicateur de distance,...). L'avantage est que l'appartenance
d'une galaxie à un groupe donné est connue à l'avance et que l'on peut donc
comparer les sélections faîtes par les algorithmes à cette "réalité". Les
sorties de codes semi-analytiques de formation de galaxies fournissent de tels
catalogues que nous avons transformés pour convenir au point de vue d'un
observateur.

Avec des mocks catalogues à notre disposition, nous avons pu tester et comparer
divers algorithmes de regroupement à un même échantillon de galaxies et avoir
une idée de leurs performances de manière quantitative et non seulement
qualitative. Nous nous sommes intéressés au plus populaire algorithme de
regroupement qu'est la méthode de la percolation ou algorithme amis d'amis
(Friends-of-Friends, FoF ci-après). Nous avons déterminé le jeu de paramètres
de liens optimums pour la sélection de groupes de galaxies avec un ensemble de
tests et de critères optimaux, que nous avons développé, pour juger de
l'efficacité d'un algorithme de groupes de galaxies. Il est également apparu
que le choix des paramètres de liens à considérer pour un FoF dépend beaucoup
de la science que l'on souhaite réaliser avec notre catalogue de groupes.

Une partie de la thèse a consisté à réaliser un tout nouvel algorithme de
regroupement nommé MAGGIE (Models and Algorithm for Galaxy Groups, Interlopers
and Environment), bayésien et probabiliste. MAGGIE utilise les a priori acquis
à l'aide des analyses des simulations cosmologiques sur les structures à
grandes échelles et les observations obtenues à partir des larges surveys sur
les galaxies pour mieux contraindre la sélection des groupes de galaxies à
partir de l'espace des phases projeté (biaisé par la distorsion des groupes
liée au décalage vers le rouge). Les résultats de la comparaison de MAGGIE avec
l'algorithme de FoF ont montré que, bien qu'équivalent dans la capacité à
retrouver les galaxies membres des groupes (complétude et fiabilité), MAGGIE
est bien meilleur dans l'estimation des propriétés des groupes (masses
stellaires, luminosités...) grâce à la probabilité d'appartenance qui réduit
l'importance des galaxies non réellement membres du groupe (interlopers).
MAGGIE réduit significativement la fraction de fausses détections de groupes de
galaxies, c'est-à-dire de groupes sporadiques, issus de la fragmentation par
les algorithmes d'un groupe réel en plusieurs sous-groupes. L'estimation de
l'environnement global est également améliorée grâce à la méthode de
correspondance d'abondance (abundance matching) qui compare et lie les
distributions des masses stellaires des galaxies centrales des groupes aux
propriétés physiques des halos de matière noire pour une meilleure précision
dans l'estimation de la masse virielle des groupes de galaxies.

Une future application de MAGGIE sur des surveys de galaxies tels que le Sloan
Digital Sky Survey ou le Galaxy and Mass Assembly, en tenant compte de tous les
problèmes liés aux observations de chacun d'eux, devrait nous permettre par la
suite d'améliorer notre compréhension des processus physiques dans les groupes
de galaxies.

% vim: set tw=79 :
