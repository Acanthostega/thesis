\cleardoublepage%
%
\begin{bartabstract}
    Galaxies lie in a large panel of environments from isolated galaxies, to
    pairs, groups or clusters. It is right to think that the environment has an
    impact on galaxy properties such as the morphology, stellar formation,
    metallicity\ldots Some studies already tried to quantify the importance of
    the global environment (linked to the dark matter halo mass) and the local
    environment (galaxy position in the group). These studies have shown that
    the environment play a minor role except for low mass galaxies. But the
    quantification of the environment is difficult since detected groups in
    redshift space (only one accessible by the observer) are very elongated and
    don't facilitate the assignment of a galaxy to a given group. If these
    quantification errors are too important, environment effects will not be
    measured correctly.

    Moreover, other physical processes are at work inside groups whose
    importance is not well understood. For example, major or minor mergers
    (rich or poor in gas, between non-central galaxies, or a central and a
    non-central by ``descent'' of its orbit by dynamical friction), rapid
    flybys harassing gas to galaxies, stripping of the interstellar gas by ram
    pressure or of gaseous reservoir by tidal forces. Although semi-analytical
    codes of galaxy formation from initial conditions of a $\Lambda$CDM
    Universe fit well observations, there are still some discrepancies possibly
    explained by a lack of environmental effects in these models.

    Our goal with these thesis is to have a detailed comprehension of the role
    of environment on galaxy properties, and finally determine the major
    physical processes in the modulation of these properties with both local
    and global environment. For this, an optimal extraction of galaxy groups
    from the projected phase space is necessary.

    We performed a study and re-implementation of some existing group finder to
    estimate their strengths and weaknesses in the detection of galaxy groups.

    A galaxy mock catalogue was created to apply several galaxy group algorithm
    on a fake galaxy sample, but with known physical properties similar to
    observations (luminosity function, density profile of galaxies in groups,
    bias in distance due to redshift). The advantage is that we know the real
    membership and we can compare the galaxy group selection to this
    ``reality''. Semi-analytical codes of galaxy formation outputs give us such
    catalogues we transformed to be coherent with the vision of an observer.

    With these mock catalogues, we tested the most popular grouping algorithm
    that is the Friends-of-Friends algorithm or percolation method. We
    determined the optimal linking lengths against the set of tests and optimal
    criterion we developed to judge the efficiency of an algorithm. It appears
    that this choice of linking lengths depends on the scientific goal to do
    with the group catalogue.

    A large part of the thesis consisted on the realization of a new grouping
    algorithm called MAGGIE (Models and Algorithm for Galaxy Groups,
    Interlopers and Environment), Bayesian and probabilistic. MAGGIE uses our
    priors acquired with analysis of cosmological simulations for large scale
    structure and of observations obtained from large galaxy surveys, to better
    constrain the selection of galaxy groups from redshift space. Comparison of
    MAGGIE with the FoF algorithm shows that MAGGIE is better in the membership
    selection (completeness, reliability), in the group properties (stellar
    mass, luminosity) with help of probabilities. The fragmentation of galaxy
    groups is also less important than FoF. Global environment estimation is
    improved by the abundance matching technique comparing and linking central
    galaxy distributions (stellar mass or luminosity) to physical properties of
    dark matter halos.

    A future application of MAGGIE on galaxy surveys such as the Sloan Digital
    Sky Survey or the Galaxy and Mass Assembly, taking care of their own
    observational problems, should improve our understanding of physical
    processes inside galaxy groups.
\end{bartabstract}

% vim: set tw=79 :
