%
\begin{bartabstract}
    \small

    Galaxies lie in a large panel of environments from isolated galaxies, to
    pairs, groups or clusters. The environment is expected to have an impact on
    galaxy properties such as morphology, stellar formation, metallicity\ldots
    Some studies already tried to quantify the importance of the global
    environment (linked to the dark matter halo mass) and the local environment
    (galaxy position in the group). These studies have shown that the
    environment play a minor role except for low mass galaxies. But the
    quantification of the environment is difficult since detected groups in
    redshift space (the only one accessible by the observer) are very
    elongated, making it difficult to extract spherical groups in real space.
    If these quantification errors are too important, environment effects will
    not be measured correctly.

    Moreover, other physical processes are at work inside groups whose relative
    roles are not well understood. For example, major or minor mergers (rich or
    poor in gas, between satellite galaxies, or after the decay of the orbit of
    a satellite onto the central galaxy by dynamical friction), rapid flybys
    harassing galaxies, stripping of the interstellar gas by ram pressure or of
    the gaseous reservoir by tidal forces. Although semi-analytical codes of
    galaxy formation from initial conditions of a $\Lambda$CDM Universe fit
    well a large set of observed relations, there are still some discrepancies
    that might be possibly by a lack of correct physical recipes of
    environmental effects in these models.

    Our goal with these thesis is to have a detailed comprehension of the role
    of environment on galaxy properties, and finally determine the major
    physical processes in the modulation of these properties with both local
    and global environment. For this, an optimal extraction of galaxy groups
    from the projected phase space is necessary.

    We performed a study and re-implementation of some existing group finder to
    estimate their strengths and weaknesses in the detection of galaxy groups.

    A galaxy mock catalogue in redshift space, designed to mimic the primary
    spectroscopic sample of the SDSS survey was created to apply several galaxy
    group algorithms. The advantage is that we know the real membership, so we
    can compare the galaxy group extracted from redshift space to the real
    space mock groups. Semi-analytical codes of galaxy formation give us such
    galaxy catalogs we transformed to be coherent with the vision of an
    observer.

    With these mock catalogues, we tested the very popular Friends-of-Friends
    grouping algorithm. We determined the optimal linking lengths against the
    set of tests and optimal criterion we developed to judge the efficiency of
    an algorithm. It appears that this choice of linking lengths depends on the
    scientific goal to do with the group catalogue.

    A large part of the thesis consisted on the realization of a new grouping
    algorithm called MAGGIE (Models and Algorithm for Galaxy Groups,
    Interlopers and Environment), Bayesian and probabilistic. MAGGIE uses our
    priors acquired with analysis of cosmological simulations for large scale
    structure and of observations obtained from large galaxy surveys, to better
    constrain the selection of galaxy groups from redshift space. Comparison of
    MAGGIE with the FoF algorithm shows that MAGGIE is superior in avoiding the
    fragmentation of real space groups, the membership selection (completeness,
    reliability) and in the group properties (group mass, luminosity). The
    better performance of MAGGIE comes from its probabilistic nature, the use
    of astrophysical and cosmological priors, and the use of halo abundance
    matching technique linking central galaxy distributions (stellar mass or
    luminosity) to physical properties of dark matter halos.

    The future application of MAGGIE on galaxy surveys such as the Sloan
    Digital Sky Survey or the deeper Galaxy and Mass Assembly, taking care of
    their own observational problems, should improve our understanding of the
    modulation of galaxy properties by their global and local environments and
    the physical processes operating inside galaxy groups.

\end{bartabstract}

% vim: set tw=79 :
