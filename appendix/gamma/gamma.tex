\bartchapterimage{heic1107a.jpg}
\chapter{Special functions}
\bartthumb{heic1107a.png}
\label{cha:gamma}

\section{Legendre elliptic integral function}

\subsection{Introduction}

The Legendre elliptic integral function appears naturally when evaluationg
distances like the luminosity distance in a flat Universe. But the most of the
time, this function is not used directly because of the difficulty in its
implementation. When it is already adapted, it is only for special values. In
following sections, we described how to use the NSWC implementation of elliptic
integrals.

\subsection{Luminosity distance}

Our goal is to easily and precisely compute the luminosity distance given a
cosmology according to the formula:

\begin{equation}
    d_L(z)=\cfrac{c(1+z)}{H_0}\int_0^z\cfrac{\dd{t}}{\sqrt{\Omega_m{(1+t)}^3+\Omega_\Lambda}}
\end{equation}

Making the change of variable $u=1/t$ and defining
$s=\sqrt{3}{(1-\Omega_m)/\Omega_m}$ gives us:

\begin{equation}
    d_L(z)=\cfrac{c(1+z)}{H_0\sqrt{s\Omega_m}}\left[T(s)-T\left(\cfrac{s}{1+z}\right)\right]
\end{equation}

with:

\begin{equation}
    T(x)=\int_0^x\cfrac{\dd{u}}{\sqrt{u^4+u}}
\end{equation}

As described in \citet{Liu+11}, this integral is an elliptic integral, and such
integrals can be expressed in terms of Carlson symmetric forms
$R_F(x_1,x_2,x_3)$:

\begin{equation}
    R_F(x_1,x_2,x_3)=
    \cfrac{1}{2}\int_0^\infty\cfrac{\dd{t}}{\sqrt{(t+x_1)(t+x_2)(t+x_3)}}
\end{equation}

With help of the reduction theorem, we can write:

\begin{equation}
    T(x)=4R_F(m,m+3+2\sqrt{3},m+3-2\sqrt{3})
\end{equation}

where:

\begin{equation}
    m(x)=\cfrac{2\sqrt{x^2-x+1}}{x}+\cfrac{2}{x}-1
\end{equation}

\section{Incomplete gamma function}
\label{sec:gamma}

\subsection{Introduction}

By definition, the incomplete gamma function $\Gamma\left({a,x}\right)$ is:

\begin{equation}
    \Gamma\left({a,x}\right)=\int_x^\infty{e^{-t}{t^{a-1}}\dd{t}}
\end{equation}

By default, many algorithms used to evaluate incomplete gamma functions do not
allowed for $a<0$. Moreover, we need to use an algorithm that do not use the
``simple'' gamma function $\Gamma\left({a}\right)$:

\begin{equation}
    \Gamma\left({a}\right)=\int_0^\infty{e^{-t}{t^{a-1}}\dd{t}}
\end{equation}
%
because that function have singularities for negative values of $a$ where $a$
is an integer, as we can see in figure \bartreffigure{gamma}.
%
\begin{figure}[hbtp]
    \centering
    \includegraphics[width=0.5\linewidth]{figures/gamma/gamma}
    \caption{The gamma function showing singularities at zero and negative
    integers.\label{fig:gamma}}
\end{figure}
%
So we need an algorithm that does not involve the gamma function for
negative values. Here is described such algorithm.

\subsubsection{Theory}

The best way to compute the incomplete gamma function for $a$ negative values
is to use recurrence relations. Let us define:
%
\begin{equation}
    \Gamma\left({a+1,x}\right)=\int_x^\infty{e^{-t}{t^{a}}\dd{t}}
\end{equation}
%
Defining $u'=e^{-t}$ and $v=t^a$, we can use integration by parts:
%
\begin{equation}
    \Gamma\left({a+1,x}\right)={\left[-e^{-t}{t^a}\right]}_x^\infty + a \int_x^\infty{e^{-t}{t^{a-1}}\dd{t}}
\end{equation}
%
The term in square brackets is always zero at infinity, $\forall a$, and the
second member of the right hand side of the previous equation lets appear the
definition of the incomplete gamma function. So the recurrence relation for the
incomplete gamma function is:
%
\begin{equation}
    \Gamma\left({a+1,x}\right)=e^{-x}{x^a} + a \Gamma\left({a,x}\right)
\end{equation}
%
We can see that computing the incomplete gamma function for $a\leqslant0$ can
be done with a recursive function using the function at higher values of $a$.
%
\begin{equation}
    \Gamma\left({a,x}\right)= \cfrac{\Gamma\left({a+1,x}\right) - e^{-x}{x^a}}{a}
\end{equation}
%
The previous equation shows that there is still a problem for integer values of
$a$ because if $a=-2$ for example, at a moment in the recursion, we have a
value of 0 for $a$ which create problems. If we refer to
\citet{abramowitz+stegun}, the definition of the elliptical integral is:
%
\begin{equation}
    E_n\left({z}\right)=\int_1^\infty{e^{-zt}}{t^{-n}}\dd{t}
\end{equation}
%
for integer values of $n$. If we change the variable in the integral to
$t'=zt$, we can rewrite the equation to have:
%
\begin{equation}
    E_n\left({z}\right)={z^{n-1}}\Gamma\left({1-n,z}\right)
\end{equation}
%
so:
%
\begin{equation}
    \Gamma\left({a,x}\right)={x^a}E_{1-a}\left({x}\right)
\end{equation}
%
for $a\leq0$ and $a$ integer. Now we have a good computation for the incomplete
gamma function. But numerically, there is still a problem near integer negative
values of $a$. If $a$ is very close to an integer value, then at some moment in
the recursion, $a$ will be very small. So $1/a$ can be greater than the
overflow value for the machine. To avoid this, we add a condition for $a$ when
it is near zero.

An other definition of the incomplete gamma function is:
%
\begin{equation}
    \Gamma\left({a,x}\right)=\Gamma\left({a}\right)-\gamma\left({a,x}\right)=\Gamma\left({a}\right)\left({1-P\left({a,x}\right)}\right)
\end{equation}
%
with:
%
\begin{equation}
    \gamma\left({a,x}\right)=\int_0^x{e^{-t}{t^{a}}\dd{t}}
\end{equation}
%
In \citet{NumericalRecipes} exists a precise computation of the function
$P\left({a,x}\right)$. We can remark that this function is not required in the
recursion if we have already access to a function that computes the incomplete
gamma function for positive values of $a$.

We provide below the algorithm for computing the incomplete gamma function
without loss of precision and without numerical problems for negative values of
$a$.
%
\subsubsection{Numerical}
%
\begin{minted}[bgcolor=griscode, linenos]{python}
def gammainc(a, x):

    """
    To compute the incomplete gamma function
    without loss of precision or without numerical
    problems. OF is the value of the overflow for
    the machine and expint( n, x ) the function
    which computes the integral function for n and x.
    """

    import numpy as np

    if x >= 0.:
        if a <= 1.:
            if a == int(a) or OF * abs(a) < 1 :
                return x ** int(a) * expint(1 - int(a), x)
            else :
                return (gammainc(a + 1, x) -
                    np.exp(-x) * (x ** a)) / a

        else :
            return gamma(a) (1 - P(a, x))
            # or call the function which computes
            # the incomplete gamma function for
            # positive values of a
\end{minted}

% vim: set tw=79 :
