\bartchapterimage{heic0911c.jpg}
\chapter{Density profiles}
\label{cha:profiles}
\bartthumb{heic0911c.png}

\section{Introduction}

In this chapter, we provide details on the computation of the density profiles
and their derived quantities. We define here the different normalizations used
along the thesis for some popular density profiles.

\subsection{Definitions}

The number of galaxies in a sphere of radius $r$ with a density profile in
number $\nu(r)$ is the case of a spherical symmetry:
%
\begin{equation}
    N\left(r\right)=\int_0^r4\pi {r'}^2 \nu(r')\dd{r'}
\end{equation}

To start, we define some dimensionless functions to facilitate the
computations.
%
\begin{eqnarray}
    N\left(r\right)&=N\left(a\right){\widetilde{N}\left(r/a\right)}\nonumber\\
    \nu\left(r\right)&=
        \cfrac{N\left(a\right)}{4\pi{a^3}}
        \,\widetilde\nu\left(r/a\right)
\end{eqnarray}
%
with $a$ the radius at which the logarithmic slope of the density profile is
equal to $-2$. We also define the same relations for a
virial normalization.
%
\begin{eqnarray}
    N\left(r\right)&={N_v}\,{\widehat{N}\left(r/\rvir\right)}\nonumber\\
    \nu\left(r\right)&=
        \cfrac{N_v}{4\pi\rvir^3}\,\widehat\nu\left(r/\rvir\right)
\end{eqnarray}

We also define the concentration $c$ as the ratio between the virial radius
$\rvir$ and the radius $a$, i.e.\ $c=\rvir/a$. We define $\rvir=r_{200}$ for
simplicity.

\section{Density profiles}
\label{sec:density_profiles}

\subsection{\citet{NFW+97}}

The NFW density profile is:
%
\begin{equation}
    \nu \left(r\right) = \cfrac{\nu_0}{r {\left(r+a\right)}^2}
\end{equation}
%
with $\nu_0$ a constant density.

We can write by integrating previous relations with $\int_0^1
x^2\widetilde\nu\dd x=\int_0^1 x^2\widehat\nu\dd x$  and searching for the
constant $\nu_0$:
%
\begin{equation}
    \widetilde\nu\left(x\right)=\cfrac{1}{\ln2-1/2}
        \quad\cfrac{1}{x{\left(1+x\right)}^2}
\end{equation}
%
\begin{equation}
    \widetilde{N}\left(x\right)=\cfrac{1}{\ln2-1/2}
        \quad\left(\ln\left(1+x\right)-\cfrac{x}{x+1}\right)
\end{equation}
%
\begin{equation}
    \widehat\nu\left(x\right)=
        \cfrac{1}{\ln\left(1+c\right)-c/ \left(1+c\right)}
        \quad\cfrac{1}{x{\left(1/c+x\right)}^2}
\end{equation}
%
\begin{equation}
    \widehat{N}\left(x\right)=
        \cfrac{1}{\ln \left(1+c\right)-c/ \left(1+c\right)}
        \quad\left(\ln\left(1+xc\right)-\cfrac{xc}{xc+1}\right)
\end{equation}

\subsection{Einasto}
\label{sub:einasto}

For an Einasto density profile:
\begin{equation}
    \nu \left(r\right) = \nu_0 \exp
    \left[- {\left(\cfrac{r}{b}\right)}^{1/m}\right]
\end{equation}

Writing the definition of the $a$ radius with this density profile, we have:
%
\begin{equation}
    {\left(\cfrac{1}{b}\right)}^{1/m} = 2m{\left(\cfrac{1}{a}\right)}^{1/m}
\end{equation}
%
leading to the following normalizations:
%
\begin{equation}
    \widetilde\nu \left(x\right) = \cfrac{{\left(2m\right)}^{3m}}{m\gamma
    \left(3m, 2m\right)} \exp \left(-2mx^{1/m}\right)
\end{equation}
%
\begin{equation}
    \widetilde{N} \left(x\right) = \cfrac{\gamma \left(3m, 2m
    x^{1/m}\right)}{\gamma \left(3m, 2m\right)}
\end{equation}
%
\begin{equation}
    \widehat\nu \left(x\right) = \cfrac{{\left(2m\right)}^{3m}}{m\gamma
    \left(3m, 2m c^{1/m}\right)} \exp \left(-2m{\left(xc\right)}^{1/m}\right)
\end{equation}
%
\begin{equation}
    \widehat{N} \left(x\right) = \cfrac{\gamma \left(3m, 2m
        {\left(xc\right)}^{1/m}\right)}{\gamma \left(3m, 2m c^{1/m}\right)}
\end{equation}

\subsection{Generalized NFW}
\label{sub:generalized_nfw}

If any previous density profiles isn't sufficient to describe the distribution
of dark matter particles or galaxies inside the halos, a solution is possibly
to fit a generalized NFW profile, whose the density is:
%
\begin{equation}
    \nu \left(r\right) =
    \cfrac{\nu_0}{r^\alpha{\left(r+a\right)}^{\beta-\alpha}}
\end{equation}

In this case:
%
\begin{equation}
    \widetilde{N} \left(x\right) = \cfrac{%
        \mathcal{B}_{-x} \left(3-\alpha, 1+\alpha-\beta\right)}
    {\mathcal{B}_{-1} \left(3-\alpha, 1+\alpha-\beta\right)}
\end{equation}
%
therefore:
%
\begin{equation}
    \widetilde\nu \left(x\right) = \cfrac{1}
    {{\left(-1\right)}^{\alpha+1}
        \mathcal{B}_{-1} \left(3-\alpha, 1+\alpha-\beta\right)}
    \quad\cfrac{1}
    {x^\alpha {\left(1+x\right)}^{\beta-\alpha}}
\end{equation}
%
For the virial normalization:
%
\begin{equation}
    \widehat{N} \left(x\right) = \cfrac{%
        \mathcal{B}_{-xc} \left(3-\alpha, 1+\alpha-\beta\right)}
    {\mathcal{B}_{-c} \left(3-\alpha, 1+\alpha-\beta\right)}
\end{equation}
%
\begin{equation}
    \widehat\nu \left(x\right) = \cfrac{1}
    {{\left(-1\right)}^{\alpha+1}
        \mathcal{B}_{-c} \left(3-\alpha, 1+\alpha-\beta\right)}
    \quad\cfrac{1}
    {{\left(xc\right)}^\alpha {\left(1+xc\right)}^{\beta-\alpha}}
\end{equation}
%
where $\mathcal{B}$ is the function defined as:
%
\begin{equation}
    \mathcal{B} \left(a, b\right) =
    \cfrac{\Gamma \left(a\right) \Gamma \left(b\right)}
    {\Gamma \left(a+b\right)} =
    \int_0^1 t^{a-1} {\left(1+t\right)}^{b-1} \dd t
\end{equation}
%
and its incomplete version is:
%
\begin{equation}
    \mathcal{B}_z \left(a, b\right) =
    \int_0^z t^{a-1} {\left(1+t\right)}^{b-1} \dd t
\end{equation}

\section{Radial velocity dispersion}
\label{sec:radial_velocity_dispersion}

Galaxies in groups (and their associated dark matter halos) are assumed to be a
system of particles only submitted to the gravitation. Neglecting mergers and
other physical processes inside galaxy groups, the number of galaxies doesn't
evolve in phase space, and the distribution function is constant along the
evolution of the system. In this case, we can use the collisionless Boltzmann
equation to extract dynamical properties of galaxy groups.

The Jeans equation is the first velocity momentum of the Boltzmann equation. In a
spherical symmetry, assuming stationarity, the Jeans equation is:
%
\begin{equation}
    \label{eq:jeans}
    \ddp{\left[\nu(r)\sigma_r^2(r)\right]}{r} +
    \cfrac{2\mybeta}{r}\left[\nu(r)\sigma_r^2(r)\right]=
    -\nu(r)\cfrac{GM(r)}{r^2}
\end{equation}
%
where $\beta \left(r\right)$ is the radial profile of velocity anisotropy
$\beta = 1 - \sigma_\theta^2/\sigma_r^2$.

We can compute the radial velocity dispersion using \bartrefequation{jeans} for
a spherical system at equilibrium. The solution to this equation is given by:
%
\begin{equation}
    \myprofil\sigma_r^2(r)=
    \int_r^\infty K_r(r,s)\nu(s)\cfrac{GM(s)}{s^2}\,\dd{s}
\end{equation}
%
with $K_r(r,s)$ the kernel of the integral defined as:
%
\begin{equation}
    K_r(r,s)=\exp\left[2\int_r^s\mybeta\cfrac{\dd{t}}{t}\right]
\end{equation}

There are two ways of normalizing the radial velocity dispersion according to
the normalization used for the density and mass profiles. We show it for the
virial normalization for illustration:
%
\begin{equation}
    \label{eq:sigma_norm}
    \widehat\sigma_r^2(x)= \cfrac{1}{\widehat\nu \left(x\right)}
    \int_x^\infty K_r(x,s)\widehat\nu(s)\cfrac{\widehat{M}(s)}{s^2}\,\dd s
\end{equation}
%
with:
%
\begin{equation}
    \sigma_r^2(r) = \cfrac{G\mvir}{\rvir}\,\widehat\sigma_r^2(r/\rvir)
\end{equation}

We are interested only in the NFW profile in the thesis, since it is accurate
enough to adjust the model. If we want an analytical form for $\sigma_r
\left(r\right)$, we need to choose a model for the anisotropy profile $\beta
\left(r\right)$. We provide here some expressions of the radial velocity
dispersion, assuming the NFW density profile, for a useful anisotropy model.

\subsection{\citet{ML+05}}
\label{sub:ml05}

This model is of the form:
%
\begin{equation}
    \beta \left(r\right) = \undemi\cfrac{r}{r+b}
\end{equation}
%
where $b$ is a characteristic radius of the model. Introducing this expression
in \bartrefequation{sigma_norm}, we obtain:
%
\begin{align}
    \widetilde\sigma_r^2(x)=&
        -\frac{1}{3 x (1+r x)
        {(-1+\ln\left(4\right))}^2}2
        \left(-\frac{1}{2}+\ln\left(2\right)\right)\nonumber\\
    &\left(x \left(3+x \left(\pi ^2 (-3+2 r)
        {(1+x)}^2-3 (-9+5 r-7 x+4 r x)
        \right)\right.\vphantom{\ln\left(1+\frac{1}{x}\right)}\right.
        \nonumber\\
    &\left.-3 x^3 \ln\left(1+\frac{1}{x}\right)+3 x (1+2 x)
        \ln\left(x\right)\right)\nonumber\\
    & -3 (1+2 x (-1-2 x (2+x)+r (1+x) (1+2 x)))
        \ln\left(1+x\right)+3 (-3+2 r) {\left(x+x^2\right)}^2
        \ln{\left(1+x\right)}^2\nonumber\\
    & \left.+6 (-3+2 r) x^2 {(1+x)}^2
        Li_2\left(-x\right)\vphantom{\ln\left(1+\frac{1}{x}\right)}\right)
        \nonumber\\
\end{align}
\begin{align}
    \widehat\sigma_r^2(x)=&
        \frac{1}{6 x (1+r x) \left(-1+\frac{1}{1+c}+\ln c\right)}
        \left(c x \left(-3+x \left(3 c \left(-9+\pi ^2\right)+
        \left(15-2 \pi ^2\right) r\right.\right.
        \vphantom{{(1+c x)}^2}\right.\\
    &\left. -4 c \left(-3+\pi ^2\right) r x+
        3 c^3 \pi ^2 x^2+c^2 x \left(-21+\pi ^2 (6-2 r x)\right)\right)\\
    & \left.+6 c^3 x^3 {\rm arccoth} \left(1+2 c x\right)-
        3 c x (1+2 c x) \ln\left(c x\right)\right)\\
    &+3(1+2 x (r+c (-1+x (-4 c+3 r+2 c (-c+r) x))))
        \ln\left(1+c x\right)\\
    &\left.+3c(3 c-2 r) x^2 {(1+c x)}^2 {\left(\ln\left(1+c x\right)\right)}^2+
    6 c (3 c-2 r) x^2 {(1+c x)}^2 \mathrm{Li_2}\left(-c x\right)\right)\\
\end{align}

\section{Line of sight velocity variance}

We will compute in this section the line of sight velocity dispersion of
galaxies in a general spherical density profile, and then compute it
specifically for an NFW profile. This is useful to make cuts at
$\pm\kappa\sigma_\mathrm{LOS} \left(R\right)$ in the \pps{}.

By definition, the variance is the mean of the squared quantity. We use a
general density profile which is invariant under rotations $\nu{(r)}$. In our
case, we make this mean on the line of sight, so:
%
\begin{equation}
    \sigma_\mathrm{LOS}^2\left({R}\right)=
    \cfrac{\int_{-\infty}^{\infty}{v_\mathrm{LOS}^2}\nu{(r)}\,\dd{z}}
    {\int_{-\infty}^{\infty}\nu{(r)}\,\dd{z}}
\end{equation}
%
But in the group, $r^2=R^2+z^2$ so:
%
\begin{equation}
    \sigma_\mathrm{LOS}^2\left({R}\right)=
    \cfrac{2\int_{R}^{r_{\max}}{v_\mathrm{LOS}^2}
    \cfrac{\nu{(r)}{r}}{\sqrt{r^2-R^2}}\,\dd{r}}
    {2\int_{R}^{r_{\max}}\cfrac{\nu{(r)}{r}}{\sqrt{r^2-R^2}}\,\dd{r}}
\end{equation}
%
The denominator is by definition the projected density surface along the line
of sight and we denote it
%
\begin{equation}
    \Sigma(R) = 2\int_{R}^{r_{\max}}\cfrac{\nu{(r)}{r}}{\sqrt{r^2-R^2}}\,\dd{r}
\end{equation}
%
Normally the integration is for $r_{\max}\rightarrow\infty$ but in our case we
want to restrict to a limited region in the group (to the virial sphere
precisely).

In the same coordinate system as previously, the line of sight velocity can be
expressed in spherical coordinates as:
%
\begin{equation}
    v_{\mathrm{LOS}} = v_r \cos\theta - v_\theta \sin\theta
\end{equation}
%
We suppose that we are at the equilibrium and so that there is no flow in the
group in consequence we can neglect means of velocities. In terms of velocity
variance we have now:
%
\begin{equation}
    \mysigma\mysiglos = 2\int_R^{r_{\max}}
    \left({\sigma_r^2(r)\cos^2\theta+\sigma_\theta^2\sin^2\theta}\right)
    \cfrac{\nu{(r)}{r}}{\sqrt{r^2-R^2}}\,\dd{r}
\end{equation}
%
If we want to use the anisotropy parameter
$\mybeta=1-\sigma_\theta^2(r)/\sigma_r^2(r)$ in case of sphericity, we can
write:
%
\begin{equation}
    \mysigma\mysiglos = 2\int_R^{r_{\max}}
    \left({1-\mybeta\cfrac{R^2}{r^2}}\right)
    \cfrac{\nu{(r)}{\sigma_r^2(r)}{r}}{\sqrt{r^2-R^2}}\,\dd{r}
\end{equation}

We can compute the radial velocity dispersion using the Jeans equation for a
spherical system at equilibrium.

\subsection{\citet{ML+05} anisotropy}

With the decomposition of the integral over the domain of integration, we can
write:
%
\begin{align}
    \Sigma{(R)}{\sigma_\mathrm{LOS}}^2{(R)}=&
        2\int_R^{r_v}{\cfrac{\left({s+a}\right)}{s^2}{\nu(s)}{G}{M(s)}}{\dd{s}}
        \nonumber\\
    &\times
        \left(\int_R^s{\left(\cfrac{r}{r+a}-\undemi
        {\left(\cfrac{R}{r+a}\right)}^2\right)
        \cfrac{1}{\sqrt{r^2-R^2}}\dd{r}}\right)\nonumber\\
    &+2\int_{r_v}^{\infty}
        \cfrac{\left({s+a}\right)}{s^2}\nu(s){G}{M(s)}\dd{s}\nonumber\\
    &\times
        \left(\int_R^{r_v}
            \left(\cfrac{r}{r+a}-\undemi{\left(\cfrac{R}{r+a}\right)}^2\right)
            \cfrac{1}{\sqrt{r^2-R^2}}\dd{r}\right)
\end{align}
%
where we are setting $r_{\max}$ to $r_v$. So now we can write:
%
\begin{align}
    {\sigma_\mathrm{LOS}}^2{(R)}=&{v_v}^2\cfrac{c/2}{\widetilde{M}{(c)}\widetilde{\Sigma}{(R/a,c)}}\nonumber\\
    &\times\left(\int_{R/a}^c{{K}\left({x\cfrac{a}{R},\cfrac{a}{R}}\right)}\widetilde{\nu}{(x)}
    \cfrac{\widetilde{M}{(x)}}{x}\dd{x}+I\left({c\cfrac{a}{R},\cfrac{a}{R}}\right){J(c)}\right)
\end{align}
%
\begin{equation}
    I(u,u_a)=\left\{\begin{array}{lr}
        -u_a{\rm{sign}}(u_a-1)\cfrac{{u_a}^2-1/2}{|{u_a}^2-1|^{3/2}}{C^{-1}\left(\cfrac{1+u{u_a}}{u+u_a}\right)}&\\
        \hspace{5em}+{\rm{acosh}}{u}+\cfrac{1/2}{u_a+u}\cfrac{\sqrt{u^2-1}}{{u_a}^2-1},&{u_a}\neq1\\
        {\rm{acosh}}{u}-\sqrt{\cfrac{u-1}{u+1}}\left(\cfrac{8+7u}{6(1+u)}\right),&{u_a}=1\\
    \end{array}\right.
\end{equation}
%
with:
%
\begin{equation}
    K(u,u_a)=\left({1+\frac{u_a}{u}}\right){I(u,u_a)}
\end{equation}
%
and:
%
\begin{equation}
    C^{-1}(X)=\left\{\begin{array}{lr}
        {\rm{acosh}}{X}&u_a>1\\
        {\rm{acos}}{X}&u_a<1\\
    \end{array}\right.
\end{equation}
%
We also have an other integral:
%
\begin{equation}
    J(y)=\int_y^{\infty}\frac{x+1}{x^2}\widetilde{\nu}{(x)}\widetilde{M}{(x)}\dd{x}
\end{equation}
%
In the case of an NFW profile, this can be expressed in an analytical way:
%
\begin{align*}
    J(y)=&
        \frac{2}{3{y^2}(1+y){\left(\ln{4}-1\right)}^2}
        \left(y\left(-3+y\left(-9+\pi^2\left(1+y\right)\right)\right)\right.\\
    &+3{y^3}\ln\left(1+\frac{1}{y}\right)+3\ln\left(1+y\right)
            \left(1-y+y^2\left(1+y\right)\ln\left(1+y\right)\right)\\
    &\left.-3{y^2}\ln\left({y}\left(1+y\right)\right)+6{y^2}
        \left(1+y\right){\dilog{-y}}\right)\\
\end{align*}
%
where the dilogarithm function is defined in our case as:
%
\begin{equation}
    \dilog{z}=-\int_0^1\cfrac{\ln{(1-zt)}}{z}\,\dd{t}
\end{equation}

For the NFW model, \citet{MBM+10} provide the expression of
$\widetilde{\Sigma}$:
%
\begin{multline}
    \widetilde{\Sigma}(X,c)=\cfrac{1}{2\ln2-1}\int_X^c\cfrac{\dd{x}}{{(1+x)}^2\sqrt{x^2-X^2}}\\
    =\cfrac{1}{2\ln2-1}\begin{cases}
        \cfrac{1}{{(1-X^2)}^{3/2}}
        \cosh^{-1}\left[\cfrac{c+X^2}{(c+1)X}\right]-
        \cfrac{1}{(c+1)}\cfrac{\sqrt{c^2-X^2}}{1-X^2} &\text{if } 0<X<1 \\
    \cfrac{\sqrt{c^2-1}(c+2)}{3{(c+1)}^2} &\text{if } X=1<c\\
    \cfrac{1}{(c+1)}\cfrac{\sqrt{c^2-X^2}}{X^2-1}-
    \cfrac{1}{{(X^2-1)}^{3/2}}
    \cos^{-1}\left[\cfrac{c+X^2}{(c+1)X}\right] &\text{if } 1<X<c\\
    0 &\text{if } X=0\,\text{or}\,X>c
    \end{cases}
\end{multline}

% vim: set tw=79 : set concealcursor=
