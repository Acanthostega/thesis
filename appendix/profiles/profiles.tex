\chapter{Density profiles}
\label{cha:profiles}

\section{Introduction}

In this chapter, there is some details on the computation of the density
profiles and there derived quantities. We define here the different
normalization used along the thesis for some useful current density
profiles.

\subsection{Definitions}

The number of galaxies in a sphere of radius $r$ with a density profile in
number $\nu(r)$ is the case of a spherical symmetry:
%
\begin{equation}
    N\left(r\right)=\int_0^r4\pi {r'}^2 \nu(r')\dd{r'}
\end{equation}

To start, we define some functions with no dimension used to make easier some
computations.
%
\begin{eqnarray}
    N\left(r\right)&=N\left(a\right){\widetilde{N}\left(r/a\right)}\nonumber\\
    \nu\left(r\right)&=
        \frac{N\left(a\right)}{4\pi{a^3}}
        \widetilde\nu\left(r/a\right)\nonumber\\
\end{eqnarray}
%
with $a$ the radius at which the slope of the density profile is equal to
-2 in logarithmic space. We also define the same relations for a virial
normalization.
%
\begin{eqnarray}
    N\left(r\right)&={N_v}{\overline{N}\left(r/\rvir\right)}\nonumber\\
    \nu\left(r\right)&=
        \frac{N_v}{4\pi\rvir^3}\overline\nu\left(r/\rvir\right)\nonumber\\
\end{eqnarray}

We also define the concentration $c$ as the ratio between the virial radius
$\rvir$ and the radius $a$, i.e.\ $c=\rvir/a$.

\section{Density profiles}
\label{sec:density_profiles}

\subsection{\citet{NFW+97}}

The definition of a NFW profile is:
%
\begin{equation}
    \nu \left(r\right) = \cfrac{\nu_0}{r {\left(r+a\right)}^2}
\end{equation}
%
with $\nu_0$ a constant.

We can write by integrating previous relations and searching for the
constant $\rho_0$:
%
\begin{equation}
    \widetilde\nu\left(x\right)=\cfrac{1}{\ln2-1/2}
        \quad\cfrac{1}{x{\left(1+x\right)}^2}
\end{equation}
%
\begin{equation}
    \widetilde{N}\left(x\right)=\cfrac{1}{\ln2-1/2}
        \quad\left(\ln\left(1+x\right)-\cfrac{x}{x+1}\right)
\end{equation}
%
\begin{equation}
    \overline\nu\left(x\right)=
        \cfrac{1}{\ln\left(1+c\right)-c/ \left(1+c\right)}
        \quad\cfrac{1}{x{\left(1/c+x\right)}^2}
\end{equation}
%
\begin{equation}
    \overline{N}\left(x\right)=
        \cfrac{1}{\ln \left(1+c\right)-c/ \left(1+c\right)}
        \quad\left(\ln\left(1+xc\right)-\cfrac{xc}{xc+1}\right)
\end{equation}

\subsection{Einasto}
\label{sub:einasto}

For an Einasto density profile:
\begin{equation}
    \nu \left(r\right) = \nu_0 \exp
    \left(- {\left(\cfrac{r}{b}\right)}^{1/m}\right)
\end{equation}

Writing the definition of the $a$ radius with this density profile, we have:
%
\begin{equation}
    {\left(\cfrac{1}{b}\right)}^{1/m} = 2m{\left(\cfrac{1}{a}\right)}^{1/m}
\end{equation}
%
leading to the following normalizations:
%
\begin{equation}
    \widetilde\nu \left(x\right) = \cfrac{{\left(2m\right)}^{3m}}{m\gamma
    \left(3m, 2m\right)} \exp \left(-2mx^{1/m}\right)
\end{equation}
%
\begin{equation}
    \widetilde{N} \left(x\right) = \cfrac{\gamma \left(3m, 2m
    x^{1/m}\right)}{\gamma \left(3m, 2m\right)}
\end{equation}
%
\begin{equation}
    \overline\nu \left(x\right) = \cfrac{{\left(2m\right)}^{3m}}{m\gamma
    \left(3m, 2m c^{1/m}\right)} \exp \left(-2m{\left(xc\right)}^{1/m}\right)
\end{equation}
%
\begin{equation}
    \overline{N} \left(x\right) = \cfrac{\gamma \left(3m, 2m
        {\left(xc\right)}^{1/m}\right)}{\gamma \left(3m, 2m c^{1/m}\right)}
\end{equation}

\subsection{Generalized NFW}
\label{sub:generalized_nfw}

If any previous density profiles isn't sufficient to describe the
distribution of dark matter particles or galaxies inside the halos, a
solution is possibly to fit a generalized NFW profile, whose the density is:
%
\begin{equation}
    \nu \left(r\right) =
    \cfrac{\nu_0}{r^\alpha{\left(r+a\right)}^{\beta-\alpha}}
\end{equation}

In this case:
%
\begin{equation}
    \widetilde{N} \left(x\right) = \cfrac{%
        \mathcal{B}_{-x} \left(3-\alpha, 1+\alpha-\beta\right)}
    {\mathcal{B}_{-1} \left(3-\alpha, 1+\alpha-\beta\right)}
\end{equation}
%
\begin{equation}
    \widetilde\nu \left(x\right) = \cfrac{1}
    {{\left(-1\right)}^{\alpha+1}
        \mathcal{B}_{-1} \left(3-\alpha, 1+\alpha-\beta\right)}
    \quad\cfrac{1}
    {x^\alpha {\left(1+x\right)}^{\beta-\alpha}}
\end{equation}
%
and for the virial normalization:
%
\begin{equation}
    \overline{N} \left(x\right) = \cfrac{%
        \mathcal{B}_{-xc} \left(3-\alpha, 1+\alpha-\beta\right)}
    {\mathcal{B}_{-c} \left(3-\alpha, 1+\alpha-\beta\right)}
\end{equation}
%
\begin{equation}
    \overline\nu \left(x\right) = \cfrac{1}
    {{\left(-1\right)}^{\alpha+1}
        \mathcal{B}_{-c} \left(3-\alpha, 1+\alpha-\beta\right)}
    \quad\cfrac{1}
    {{\left(xc\right)}^\alpha {\left(1+xc\right)}^{\beta-\alpha}}
\end{equation}
%
where $\mathcal{B}$ is the function defined as:
%
\begin{equation}
    \mathcal{B} \left(a, b\right) =
    \cfrac{\Gamma \left(a\right) \Gamma \left(b\right)}
    {\Gamma \left(a+b\right)} =
    \int_0^1 t^{a-1} {\left(1+t\right)}^{b-1} \dd t
\end{equation}
%
and its incomplete version is:
%
\begin{equation}
    \mathcal{B}_z \left(a, b\right) =
    \int_0^z t^{a-1} {\left(1+t\right)}^{b-1} \dd t
\end{equation}

\section{Radial velocity dispersion}
\label{sec:radial_velocity_dispersion}

Galaxies in groups (and their associated dark matter halos) are assumed to
be a system of particles only submitted to the gravitation. Neglecting
mergers and other physical processes inside galaxy groups, the number of
galaxies doesn't evolve in phase space, and the distribution function is
constant along the evolution of the system. In this case, we can use the
Boltzmann equation without collisions to extract dynamical properties of
galaxy groups.

The Jeans equation is a particular case of the Boltzmann equation, assuming
a spherical symmetry of the system and stationarity. It is expressed as:
%
\begin{equation}
    \label{eq:jeans}
    \ddp{\left(\nu(r)\sigma_r^2(r)\right)}{r} + \cfrac{2\mybeta}{r}\left(\nu(r)\sigma_r^2(r)\right)=
    -\nu(r)\cfrac{GM(r)}{r^2}
\end{equation}

We can compute the radial velocity dispersion using \bartrefequation{jeans}
for a spherical system at equilibrium. The solution to this equation is
given by:
%
\begin{equation}
    \myprofil\sigma_r^2(r)=\int_r^\infty K_r(r,s)\nu(s)\cfrac{GM(s)}{s^2}\dd{s}
\end{equation}
%
with $K_r(r,s)$ the kernel of the integral defined as:
%
\begin{equation}
    K_r(r,s)=\exp\left[2\int_r^s\mybeta\cfrac{\dd{t}}{t}\right]
\end{equation}

There is two ways of normalizing the radial velocity dispersion according to
the normalization used for the density and mass profiles. We show it for the
virial normalization for illustration:
%
\begin{equation}
    \label{eq:sigma_norm}
    \overline\sigma_r^2(x)= \cfrac{1}{\overline\nu \left(x\right)}
    \int_x^\infty K_r(x,s)\overline\nu(s)\cfrac{\overline{M}(s)}{s^2}\dd s
\end{equation}
%
with:
%
\begin{equation}
    \sigma_r^2(r) = \cfrac{GM_v}{\rvir}\overline\sigma_r^2(r/\rvir)
\end{equation}

We are interested only in the NFW profile in the thesis, since it
is accurate enough to adjust the model. If we want analytical form, we need
to choose the model we want for the anisotropy profile. We provide here some
expressions of the radial velocity dispersion, assuming the NFW density
profile, for some anisotropy models.

\subsection{\citet{ML+05}}
\label{sub:ml05}

This model is of the form:
%
\begin{equation}
    \beta \left(r\right) = \undemi\cfrac{r}{r+b}
\end{equation}
%
where $b$ is a characteristic radius of the model. Introducing this
expression in \bartrefequation{sigma_norm}, we obtain:
%
\begin{align}
    \widetilde\sigma_r^2(x)=&
        -\frac{1}{3 x (1+r x)
        {(-1+\ln\left(4\right))}^2}2
        \left(-\frac{1}{2}+\ln\left(2\right)\right)\nonumber\\
    &\left(x \left(3+x \left(\pi ^2 (-3+2 r)
        {(1+x)}^2-3 (-9+5 r-7 x+4 r x)
        \right)\right.\vphantom{\ln\left(1+\frac{1}{x}\right)}\right.
        \nonumber\\
    &\left.-3 x^3 \ln\left(1+\frac{1}{x}\right)+3 x (1+2 x)
        \ln\left(x\right)\right)\nonumber\\
    & -3 (1+2 x (-1-2 x (2+x)+r (1+x) (1+2 x)))
        \ln\left(1+x\right)+3 (-3+2 r) {\left(x+x^2\right)}^2
        \ln{\left(1+x\right)}^2\nonumber\\
    & \left.+6 (-3+2 r) x^2 {(1+x)}^2
        Li_2\left(-x\right)\vphantom{\ln\left(1+\frac{1}{x}\right)}\right)
        \nonumber\\
\end{align}
\begin{align}
    \widetilde\sigma_r^2(x)=&
        \frac{1}{6 x (1+r x) \left(-1+\frac{1}{1+c}+\ln\left(c\right)\right)}
        \left(c x \left(-3+x \left(3 c \left(-9+\pi ^2\right)+
        \left(15-2 \pi ^2\right) r\right.\right.
        \vphantom{{(1+c x)}^2}\right.\\
    &\left. -4 c \left(-3+\pi ^2\right) r x+
        3 c^3 \pi ^2 x^2+c^2 x \left(-21+\pi ^2 (6-2 r x)\right)\right)\\
    & \left.+6 c^3 x^3 {\rm arccoth} \left(1+2 c x\right)-
        3 c x (1+2 c x) \ln\left(c x\right)\right)\\
    &+3(1+2 x (r+c (-1+x (-4 c+3 r+2 c (-c+r) x))))
        \ln\left(1+c x\right)\\
    &\left.+3c(3 c-2 r) x^2 {(1+c x)}^2 {\left(\ln\left(1+c x\right)\right)}^2+
        6 c (3 c-2 r) x^2 {(1+c x)}^2 Li_2\left(-c x\right)\right)\\
\end{align}
