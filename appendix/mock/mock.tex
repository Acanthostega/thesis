\bartchapterimage{simudm}
\bartthumb{thumb_simudm}
\cs{Generate mock catalogues}

% une mini table des matières
%\minitoc%

\fs{Introduction}

\comments{Add things to this section!}A mock catalogue is a useful tool to test algorithms involving galaxies in order to see if it
is operational in a realistic situation. Many of the properties of galaxy surveys can be simulated: the spatial clustering of
galaxies, luminosity function, incompleteness and measures errors are some examples of them. There are different methods to obtain
such a mock catalogue. All of them involves cosmological simulations and there halos of dark matter. According to the model of
galaxy formation, we can use halo occupation distribution (HOD) to populate dark matter haloes with galaxy and putting some LF as
constraint. We can too follow galaxies in semi analytical models (SAM) in those cosmological simulations outputs in order to have
statistical properties of galaxies which agree with observational results. Then with such realistic galaxies we can use those
simulation boxes to place an observer into it and create a mock survey. But to have a realistic mock catalogue, it's necessary to
take care of many things which will be described in the next section.

\fs{Mock structure}

In all this section, we will assume that we have already in our possession a dark matter simulation box which has been populated
with galaxies with one of the methods described below (SAM, HOD...). At this step, physical properties of those galaxies aren't
interesting.

\fss{Placing boxes}

The first step to make a mock catalogue is to get galaxies positions like in a survey, to get an $(\alpha,\delta)$ frame to simulate
the sky coverage of survey and, at the same time, project galaxies on the sky, masking to us some spatial modulations of galaxy
properties.

We want that a false observer see the same volume extension that a true survey. For example for the SDSS survey, we can measure
redshift to a value of 0.3 (and more !). But the problem is that the majority of the simulation boxes have a size of around
$L_{\mathrm{box}}=\num{100}-\num{300} h^{-1}$ Mpc, letting us with a maximal redshift in our false survey of around
${H_0}{L_{\mathrm{box}}}/c\approx\num{0.025}$ in the case of a box of $\num{100} h^{-1}$ Mpc sized. Bigger simulations exist, and
maybe can allow us to access to bigger redshifts, but this increasing size reduces the resolution of the simulation in particle mass
and therefore we can't have low mass halos in the simulations.

The solution is to take a "little" simulation box and to replicate it and to make some bigger "Tetris" cube until we reach the
maximal redshift we want. An example of the resulting "mock cube" is shown on figure (\ref{fig:cubemock}).
\begin{wrapfigure}{l}{0.4\linewidth}
    \centering
    \includegraphics[width=\linewidth]{mock}
    \caption{\footnotesize{}The structure of the mock catalog once we have replicated the simulation box chosen to populate dark
    matter halos.}%
    \label{fig:cubemock}
\end{wrapfigure}

Now if we take an observer at some position into this big box, we can have different sky coverage for the observer. The simplest is
to place the observer at a corner which we give a solid angle of $\pi/2$ steradians. At the centre, we have a full sky coverage but
we reduce the redshift extension by 2.

At this time we don't care about a redshift evolution of galaxies for the observer, but if we want to care about it, we need to use
other snapshots at different redshifts. Box sizes are similar in comobile coordinates, and different in physical coordinates
due to the variation of the Hubble constant with redshift ($h$ depends en $z$). So juxtaposing cubes isn't as easy as the case
without redshift evolution.\comments{Maybe add details if I use it later...}

Many simulations give coordinates in units of $h^{-1}\mathrm{Mpc}$ but we place galaxies in the mock in physical positions which we
can measure. So coordinates in the mock catalog are scaled to get positions in units of $\mathrm{Mpc}$.

\begin{wrapfigure}{r}{0.3\linewidth}
    \centering
    \includegraphics[width=\linewidth]{cube}
    \caption{Consequences of a random rotation angle on initial cube.}
\label{fig:rotation}
\end{wrapfigure}

Placing boxes as described previously can create a perspective effect from the point of view of an observer, and the consequences
aren't predictable in a statistical sense when we try to use the mock catalog. To avoid this, we apply some transformations on
galaxies in the initial cube like inversions, rotations and periodic translations. Rotations are multiples of $\pi/2$ around the
three principal coordinates axes, because if other rotations are allowed, it can create some over-densities in some regions of the
final mock which aren't physical. An example of such a problem is illustrated in figure (\ref{fig:rotation}).
Translations are performed on the three principal axes and when galaxies are out of the initial cube, periodic conditions are
applied. All of those transformations are randomly generated for each cube in the final mock catalog.

\fss{Physics}
We have now galaxies in a realistic coverage of the Universe, restricted to a given volume. But an observer can't use it because
galaxies are seen in projection and he doesn't have any idea of the real distance of galaxies from him in a statistical sense. So we
need to add physic and errors to those galaxies.

\fsss{Survey mask}

When studying galaxies in a survey, we are confronted to the problem of the area of this survey. Limits are not defined in a easy
analytical way in many cases which can create some problems to correct for observers. The first step to simulate this is to
transform cartesian coordinates in the 3D space to celestial coordinates ($(\alpha,\delta)$ frame). Get these coordinates is the
same as to compute spherical coordinates.
\begin{eq}
        \alpha=\left\{ \begin{array}{lcr}
                         \mbox{arctan2}(Y,X)+2\pi & \mbox{if} & Y>0 \\
                         \mbox{arctan2}(Y,X) & \mbox{else} & \\
                        \end{array}\right.\nonumber
\end{eq}
\begin{eq}
        \delta=\mbox{sign}(Z)\arccos\left(\frac{\sqrt{X^2+Y^2}}{\sqrt{X^2+Y^2+Z^2}}\right)
\end{eq}
In our case, the origin of coordinates is the observer. If we keep the distance as calculated previously, the observer can still
have precise determination of the distance of a galaxy. In reality, we observe it in redshift space so the redshift as distance
indicator is biased by peculiar velocities. Our initial galaxy catalog allow us to get the velocity of a galaxy, so we can compute
the line of sight (los) velocity of this galaxy relatively to the observer.
\begin{eq}
	v_{\mathrm{los}}=\cfrac{\vec{OG}.\vec{v_{\mathrm{pec}}}}{||\vec{OG}||}
\end{eq}
where $O$ is the observer and $G$ the galaxy, $\vec{v_{\mathrm{pec}}}$ its peculiar velocity. This velocity has a sign. The redshift
is just the expression a shift in wavelength due to a velocity. The observed wavelength $\lambda$ is linked to the original
(emitted) wavelength $\lambda_0$ by:
\begin{eq}
	\lambda=(1+z)\lambda_0
\end{eq}
The shift caused by Universe expansion is $\lambda_{\mathrm{cos}}=(1+z_{\mathrm{cos}})\lambda_0$ where the subscript $\emph{cos}$
refer to the cosmological expansion. The shift caused by the peculiar velocity is
$\lambda=(1+z_{\mathrm{pec}})\lambda_{\mathrm{cos}}$. So the observed wavelength is
$\lambda=(1+z_{\mathrm{pec}})(1+z_{\mathrm{cos}})\lambda_0$. The resulting observed redshift is just:
\begin{eq}
	(1+z)=(1+z_{\mathrm{pec}})(1+z_{\mathrm{cos}})
\end{eq}
expansion. The peculiar redshift is the just due to the relativist Doppler effect:
\begin{eq}
	(1+z_{\mathrm{pec}})=\sqrt{\cfrac{1+\beta}{1-\beta}}
\end{eq}
with $\beta={v_{\mathrm{los}}}/{c}$. The cosmological redshift is approximated by $z_{\mathrm{cos}}={H_0}{D}/c$ where $D$ is the
physical distance of the galaxy to the observer and $H_0$ the Hubble constant. \comments{I think we need to add the velocity of the
Local Group in the redshift, because in our case the observer has a null velocity. Maybe corrected in SDSS data ?}Applying this
method to the galaxies in the mock catalog, we can have galaxies whose distance is biased by peculiar velocities in redshift space.
With such a treatment, the velocity dispersion of galaxies in groups leads to the apparition of "fingers of God" as seen in
observations in redshift space.

With our frame in redshift space relative to the observer, we can apply different masks on angular coordinates according to the
survey we want to mimic. An example of such a mask is in appendix (\ref{ap:sdss}).

\fsss{K-corrections}

In reality, an observer study galaxies in a given bandwidth in wavelength and can't use the bolometric flux of the object. With the
expanding Universe, all the spectral energy distribution (SED) of galaxy is shifted. All wavelengths are shifted by the same value
for a given redshift. So, knowing the luminosity $L$ of a galaxy in a given band in reality (using the true SED), computing its
apparent magnitude for an observer isn't as easy as correcting for the distance modulus. The observer in the same band sees a
different part of the true SED. The flux observed in the same band as the true flux is maybe higher or lower. A correction for this
effect is needed and must be taken into account in our mock catalogue.

As explained before, this correction depends on the SED of galaxies and the band used in the survey. The common way of correcting it
when we have a multi-band photometry is to fit the observed SED in those bands with theoretical templates of SEDs. Such templates
can be obtained with existing programs as PEGASE \comments{Add references}, which give us SEDs with some assumptions on the galaxy.
But those programs are a little time consuming, which can be a problem for mock when we want to run several of them. A good solution
is given by \citet{CMAZ10}, where the K-correction is fitted on templates for SED as given by PEGASE in terms of a polynomial of
the redshift of the galaxy and its colour. The corresponding K-correction is precise for redshifts until 0.3 in different survey
bands (including $ugriz$ for the SDSS). This work reduces the computation of K-corrections to the use of simple polynomial relations
and make easier our task.

By definition, the K-correction $K$ for a galaxy of apparent magnitude $m_X$ in a given band $X$ and absolute magnitude $M_X$ in the
same band is:
\begin{eq}
	{m_X}={M_X} + {5\log_{10}\pg{d_{\mathrm{lum}}\left[pc\right]}\pd} - 5 + K
\end{eq}
In our case, the K-correction depends on the redshift of the galaxy and its colour in apparent magnitude given two bands. So we can
rewrite:
\begin{eq}\label{eq:appmag}
	m_X = M_X + 5\log_{10}\pg{d_{\mathrm{lum}}\left[pc\right]}\pd - 5 + K( z, m_X - m_{X'} )
\end{eq}
where:
\begin{eq}
        K(z,m_{X}-{m}_{X'})=\sum_{i=0}^{N_i}\sum_{j=0}^{N_j}{a_{ij}}{z^i}{(m_X-{m}_{X'})^j}
\end{eq}
and $a_{ij}$ is a ${N_i}\times{N_j}$ matrix containing the coefficients of the two dimensional polynomial. These coefficients depend
on the bands of the survey used for the colour computation.

The observer in the mock can just, in theory, access to apparent magnitude of the survey. But we don't know in advance these
magnitudes, and as we can see in the expression of equation (\ref{eq:appmag}), we need apparent magnitudes to compute apparent
magnitudes. If we use the other bands of the survey, with $a_{ij}$ coefficients, we can always write a set of equations for a galaxy
which involves all apparent magnitudes of the survey. So we can write a set of non linear equations with polynomial of order $N_j$
(redshift of the galaxy is supposed to be known). Numerically it's easy to solve this set of equations, and relatively fast with
equations solvers or by iterations. In practice, the first is faster than the second method, unless both methods give similar
results in apparent magnitudes.

\fsss{Flux limit}

We have seen in appendix (\ref{ap:sdss}) that spectroscoped galaxies are just defined for galaxies whose apparent magnitude is less
than \num{17.77} in $r$ band. So, in all the redshift sample, we miss some galaxies not sufficiently bright. To take into account
this effect, we remove galaxies which don't reach the limit apparent magnitude of the survey.\comments{Some surveys have variation
of this flux limit with the sky position. Maybe something like this to check for SDSS?}

\fsss{Spectroscopic and photometric redshifts}

Sometimes, we can't access to spectroscopic redshifts which are more precise than photometric redshifts. In the SDSS, for example,
this is due to tiling process. Fibers analysing the spectrum of galaxies can't be closer from each other than \num{55}'', so if for
a target galaxy (selected to get a spectrum) there is an other galaxy closer than those 55'', the tile containing all fibers doesn't
have the possibility to measure the redshift of this galaxy. This problem is more significant for dense regions in the celestial
plane. A very good algorithm to placing tiles in order to limit the number of missed galaxies (\textit{ie} the number of fiber
collision) has been applied in the galaxy sample of the SDSS. But there is still some galaxies without spectroscoped redshifts. If
we remove those galaxies from our sample, there will be a spectroscopic incompleteness with unknown effects on our results.

To "correct" this problem, non-spectroscoped galaxies will be affected a photometric redshift in our mock catalogue in the sense
that we will affect a redshift following a normal distribution modulate mean and dispersion depending the distance of the nearest
spectroscoped neighbour and the magnitude difference between them. It is applied randomly in a given number of galaxies in the mock
catalogue according to the fraction of non-spectroscoped galaxies in the survey we want to mimic.

\fsss{Observational errors}

Errors exist in redshift measurement, photometry, astrometry, etc... We add them in our mock catalogue by simply assuming a
distribution for errors around the true value, and then applying it to the galaxy in our mock catalogue.\comments{Think in a way of
adding the surface brightness selection criterion on the mock catalogue. It will be great !}.
