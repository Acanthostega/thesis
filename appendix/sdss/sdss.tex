\bartchapterimage{sdss}
\bartthumb{thumb_sdss}
\chapter{Analysing SDSS-DR8\label{ap:sdss}}
%
\section{Introduction}
%
In order to realize the mock catalogue for the group finder algorithm, we need
to mimic the SDSS\@. This mock have to be realistic. But our algorithm needs to
have the redshift for \emph{all} galaxies in the volume selected and the SDSS
provides spectroscopic redshifts just for galaxies that could have been
targeted due to the problem of fibre collision. So for galaxies in this
situation we use the photometric redshift. The problem is that dense regions on
the survey are more susceptible to don't have a spectroscopic redshift than a
galaxy in a lower dense region. So we have to determine how the fraction of
photometric galaxies depends on the density of galaxies in the sky in order to
apply this in our mock catalogue.

In the SDSS, there are different ways to estimate photometric so we list the
methods here. \com{Add the list of method available in the SDSS}.

We can select galaxies in the sample with an SQL query. All queries used will
be summary here.
%
\begin{wrapfigure}{r}{0.5\linewidth}
    \centering
    \includegraphics[width=\linewidth]{figures/sdss/SpectroGalStripesSDSSoriginal.png}
    \caption{Spectroscoped galaxies in the SDSS DR8 with stripes limits as
    planned. Coordinates are in degrees.}
\label{fig:SDSSspecgalstripes}
\end{wrapfigure}
%
\section{Analysis}
%
\subsection{Definitions}
%
In the SDSS there is something called ``stripes'' which is a band of
observations in the sample. Those \emph{bands} can overlap contrary to
``chunks'' which are similar bands but don't overlapping (they make a complete
partition of the survey in their union). We can use this stripes in order to
select galaxies in regions of interest for our studies. Data on the SDSS
provide limits of this stripes, so we can use it to fix borders of the survey.
In reality, in the region of the survey we consider, we don't see overlapping
of the stripes. So it is more useful to use them in order to define limits of
survey in region of our interest.

Following definitions given in the SDSS website, we can define two other
coordinate systems in the survey which we can use to select galaxies.
\begin{description}
    \item[Great Circle:] This coordinates system is define with two angle
        $(\mu, \nu)$. Coordinates are relatives to one stripe so it can be use
        when working with galaxies in the region of the stripe we consider.
        \com{More definitions of this}.

    \item[Survey Coordinates:] It's an other system similar to celestial
        coordinates but ``centred'' on the ``block'' of galaxies of the survey that
        we can see in maps. Coordinates are written $(\eta,\lambda)$. If we use
        celestial coordinates, we have:
    \begin{equation}
            {(0,90°)}_{(\eta,\lambda)}={(275°,0)}_{(\alpha,\delta)}\;\;\;\;
            {(57.5°,90°)}_{(\eta,\lambda)}={(0,90°)}_{(\alpha,\delta)}
    \end{equation}
    It results from this that $-\cfrac{\pi}{2}<\eta<\cfrac{\pi}{2}$ and $-\pi<\lambda<\pi$.
\end{description}

Wee this informations we can write the transformations between the different
coordinate systems.
%
\subsubsection{Survey coordinates to celestial coordinates}
%
From previous definitions, we see that the relation between those systems is
just a rotation. So:
\begin{eqnarray}
        \delta &=& \arcsin\pg\cos\lambda\sin\pg\eta+32.5°\pd\pd\nonumber\\
        \alpha &=& \mathrm{atan2}\pg\sin\lambda,\cos\lambda\cos\pg\eta+32.5°\pd\pd+185°\nonumber\\
\end{eqnarray}
%
\subsubsection{Celestial coordinates to survey coordinates}
%
The inverse transformation is in consequence:
\begin{eqnarray}
        \eta &=& \mathrm{atan2}\pg\sin\delta,\cos\delta\cos\pg\alpha-\alpha_0\pd\pd-\delta_0\nonumber\\
        \lambda &=& \arcsin\pg\cos\delta\sin\pg\alpha-\alpha_0\pd\pd\nonumber\\
\end{eqnarray}
with ${(\alpha_0,\delta_0)}_{(\alpha,\delta)}={(0,0)}_{(\eta,\lambda)}$. We
have to apply too periodic conditions in the angles founded by the latter
equation in order to have values in the correct range. So conditions are:
\begin{eqnarray}
        \mathrm{Where}&\;& \eta<-90°\;\mathrm{or}\; \eta>90°:\nonumber\\
        & & \eta\rightarrow\eta+180°\nonumber\\
        & & \lambda\rightarrow180°-\lambda\nonumber\\
\end{eqnarray}
\begin{eqnarray}
        \mathrm{Where}&\;& \eta>180°:\nonumber\\
        & & \eta\rightarrow\eta-360°\nonumber\\
\end{eqnarray}
\begin{eqnarray}
        \mathrm{Where}&\;& \lambda>180°:\nonumber\\
        & & \lambda\rightarrow\lambda-360°\nonumber\\
\end{eqnarray}

Determining the number of a stripe to which a galaxy pertains is easy too because stripes are organized such they have a constant
width along the $\eta$ coordinate, with a width of 2.5°. The number of the stripe $n$ of a galaxy with $\eta$ position is:
\begin{equation}
        n = \mathrm{floor}\pg\cfrac{\pg\eta+58.75°\pd}{2.5°}\pd
\end{equation}

\subsection{Galaxies selection}
There are many tables in the SDSS saving galaxies and other objects properties
extracted from images of the survey. Those tables are the results of different
selections in objects detected in images. When crossing objects between images
of the survey that overlap, there are some differences of positions between the
same object in the two images. So there are possibilities that an object is
observed twice or more. In many of those tables, there is no ``double objects''.

The \texttt{Galaxy} view is a selection from the \texttt{PhotoPrimary} for
objects flagged as \emph{galaxy}. The \texttt{Galaxy} view contains the
photometric parameters (no redshifts or spectroscopic parameters) measured for
resolved primary objects. But we have other useful informations to link with
tables that give us photometric and spectroscopic redshifts. There is the
\texttt{specobjid} to link with spectroscopic redshifts in the table
\texttt{SpecObj} which doesn't contain duplicates (it's a clean table of
\texttt{SpecObjAll} with clean redshifts). If \texttt{specobjid=0}, the galaxy
doesn't have a spectroscopic redshift. The \texttt{objid} is a link to the
\texttt{Photoz} table which contains all photometric redshifts for galaxies in
the \texttt{Galaxy} table. Estimation is based on a robust fit on
spectroscopically observed objects with similar colors and inclination angle.
There is also the \texttt{PhotozRF} where estimates are based on the Random
Forest technique. Galaxies in the \texttt{SpecObj} are limited to $m_r<17.77$
and a surface brightness selection \com{Add This!!}. So we need to do the same
flux limitations when selecting galaxies on the \texttt{Galaxy} table. A
possible SQL query for selecting galaxies in this table and link them with
redshifts tables could be for spectroscoped galaxies:
\begin{listing}[H]
\begin{minted}[bgcolor=griscode, linenos]{sql}
select GG.ra, GG.dec, GG.petroMag_u, GG.petroMag_g, GG.petroMag_r,
GG.petroMag_i, GG.petroMag_z, GG.specobjid, GG.objid, Z.z, Z.Zerr
from Galaxy as GG, SpecObj as Z
where Z.specobjid=GG.specobjid and GG.specobjid!=0 and GG.petroMag_r<17.77
and GG.ra<275 and GG.ra>100 and GG.dec>-10 and GG.dec<75
\end{minted}
\end{listing}
\noindent and for galaxies which couldn't be spectroscoped:
\begin{listing}[H]
\begin{minted}[bgcolor=griscode, linenos]{sql}
select GG.ra, GG.dec, GG.petroMag_u, GG.petroMag_g, GG.petroMag_r,
GG.petroMag_i, GG.petroMag_z, GG.specobjid, GG.objid, Z.z, Z.Zerr
from Galaxy as GG, Photoz as Z
where GG.specobjid=0 and GG.objid=Z.objid and GG.petroMag_r<17.77
and GG.ra<275 and GG.ra>100 and GG.dec>-10 and GG.dec<75
\end{minted}
\end{listing}

Limits of stripes are given in the SDSS table \texttt{StripeDefs} but this
limits aren't actual limits, they are planned limits when survey started. We
can see it on the figure (\ref{fig:SDSSspecgalstripes}) where planned limits
are shown in red and spectroscoped galaxies are the points.

We see that some planned regions aren't still observed (spectroscopically
speaking). So we need to define other limits in $\lambda$ coordinates for that
stripes that aren't completes. We find by hand the new limits of stripes which
contains spectroscoped galaxies. Now, the survey mask is like in figure
(\ref{fig:SDSSspecgalstripesnew}). We will consider just galaxies in this mask
in order to find groups in the SDSS\@. Other galaxies aren't easy in order to
define borders of the survey and find groups.
\begin{figure}[ht]
    \centering
    \includegraphics[width=\linewidth]{figures/sdss/SpectroGalStripesSDSSnew.png}
    \caption{Spectroscoped galaxies in the SDSS DR8 with stripes
    limits chosen in order to find easily groups at the border of the survey.}
\label{fig:SDSSspecgalstripesnew}
\end{figure}

For fibre collisions galaxies, we use galaxies selected in the table of the
photometric redshifts and keep galaxies that are in the mask defined
previously. Now we have a sample of galaxies in a region of the SDSS for which
we can easily characterize borders and where all galaxies, given the flux limit
of the SDSS, are presents. There is just the problem of fibre collisions
galaxies for which the redshift in our possession is photometric, in
consequence less precise than spectroscopic redshifts. But our algorithm is
tested on a mock catalogue which is ``perfect'' if we don't take in account
this problem of less robust photometric redshifts. In order to know the
behaviour of the algorithm with those problematic redshifts, we need to
implement this in our mock catalogue.
%
\subsubsection{Flags in the SDSS}
%
Galaxies can have some troubles with photometry due to fit and estimations in
the SDSS\@. in the general case, those objects are flagged with the
\texttt{clean} property which indicates by 1 that the photometry is OK
and by 0 when there is a problem. Details of the problems are in the bit
flag. But for groups, we need to select all galaxies, whether there are not
clean.

\texttt{Galaxy} table is a selection from \texttt{PhotoPrimary} view for
objects with $\mathrm{\texttt{type}}=3$ (galaxy). I think that we don't have to
care of the ``good'' photometry of galaxies in the \texttt{Galaxy} view, but we
can leave a flag in the group finder algorithm to say if a galaxy is in this
case.

However, we have to take into account the error on the redshift estimation
using the \texttt{zErr}. For photometric redshift I think that if the
\texttt{zErr} is too high, we can use the \texttt{nnAvgZ} which is the average
redshift of galaxies in the neighbourhood of the considered galaxy. It can be
better too if the photometric redshift is too different from it.

The \texttt{SpecObjAll} contains duplicates and bad datas. But the
\texttt{SpecObj} contains just clean spectras. We use \texttt{zWarning} to
decide if we keep the redshift (\texttt{zWarning}=0) or not. In the latter
case, we use the photometric redshift instead.
%
\subsection{Fibre collision estimation}
%
In the SDSS, obtaining spectroscopic redshifts of galaxies is done using a
plate of 1.5° diameter, in which there is a certain number of fibres in
order to get spectrum of the galaxy. But in the field of the plate, the number
of fibres is limited, and the number of coverings of a portion of the sky is
limited too because of the time needed to obtain a spectrum. Although runs may
overlap, there is sometime galaxies that can't be spectroscoped. Moreover,
fibres have a dimension of 55'', so when galaxies are closer than this
size, one (or more) of those galaxies aren't spectroscoped. We can see that in
the figure (\ref{fig:angreddist}) where we have taken the nearest neighbour of a
galaxy and determined the differences in angular size and redshift between the
two galaxies. As expected, the number of galaxies which are closer than
55'' decreases dramatically. There are still some galaxies because the
overlapping of runs can permit to get redshifts for galaxies behind this limit.
\begin{figure}[ht]
    \centering
    \includegraphics[width=0.6\linewidth]{figures/sdss/distredSDSSDR8}
    \caption{\footnotesize{}Distribution of spectroscoped galaxies in the SDSS DR8 in angular size and redshift differences with
    the nearest neighbour galaxy.}
\label{fig:angreddist}
\end{figure}

A consequence of those problems is that in denser regions, the number of fibre
collision increases, affecting more our groups analysis because the number of
photometric redshifts is higher in those dense regions.

We need to implement this selection effect in our mock catalogue. For that we
compute the local density in the field, taking all galaxies in the
neighbourhood of 1.5° of a galaxy, and in the same time, we determine the
fraction of galaxies that don't have a spectroscopic redshift. We deduce of
this a relation between the density field and the fraction of fibre collisions.
In the mock catalogue, we compute the same density field and we apply the
relation estimated in the SDSS sample to the mock.\com{Do it!!} We need for
each galaxy to count the fraction of non spectroscoped galaxies in a region of
1.5° radius around. We have to remove galaxies that are to close of the
border of the survey, because if we don't remove those galaxies, there are some
regions with missing galaxies and the fraction estimation will be affected. The
way of selecting those galaxies is to compute a circle of 1.5° around a
galaxy, and if a generated point is out of the survey, the galaxy is defined as
to be closer to the limits.

We can generate samples of points at an angular distance $d$ to a point of
coordinate $(\alpha_0,\delta_0)$ using formulas of the spherical triangle. If
we define a triangle by the pole, the point $(\alpha_0,\delta_0)$ and the point
whose we want coordinates $(\alpha,\delta)$ denoted $M$, we can write the
following relations:
%
\begin{eqnarray}
        \sin\pg\alpha-\alpha_0\pd&=&\cfrac{\sin{d}\sin\gamma}{\cos\delta}\nonumber\\
        \sin\delta_0\cos\gamma&=&\cos\delta_0\cot{d}-\sin\gamma\cot\pg\alpha-\alpha_0\pd\nonumber\\
\end{eqnarray}
%
where $\gamma$ is like a polar angle, which have all the values between 0 and $2\pi$.
So we have now:
%
\begin{eqnarray}
        \alpha-\alpha_0 &=& \arctan\pg\cfrac{\sin\gamma}{\cos\delta_0\cot{d}-\sin\delta_0\cos\gamma}\pd\nonumber\\
        \delta &=& \arccos\pg\cfrac{\sin{d}\sin\gamma}{\sin\pg\alpha-\alpha_0\pd}\pd\nonumber\\
\end{eqnarray}
%
There are problems in poles and equator with those formulas. For a $\gamma$
limit, angles can't be recovered with those formulas. We have in those cases:\\
with $\gamma_0=\arccos\pg\cfrac{-\sin\delta_0\cos{d}}{\cos\delta_0\sin{d}}\pd$
%
\begin{eqnarray}
        \mathrm{Where}&\;& \delta_0-d<0\; \mathrm{and} \;\delta_0>0 \; \mathrm{and}\; \gamma_0<\gamma<2\pi-\gamma_0:\nonumber\\
        & & \delta\rightarrow-\delta\nonumber\\
\end{eqnarray}
%
\begin{eqnarray}
        \mathrm{Where}&\;& \delta_0+d>0\; \mathrm{and}\;\delta_0<0 \; \mathrm{and}\; \gamma_0>\gamma\;\mathrm{or}\;\gamma>2\pi-\gamma_0:\nonumber\\
        & & \delta\rightarrow-\delta\nonumber\\
\end{eqnarray}
%
with $\gamma_0=\arccos\pg\cfrac{\cos\delta_0\cot{d}}{\sin\delta_0}\pd$
%
\begin{eqnarray}
        \mathrm{Where}&\;& \delta_0+d>\cfrac{\pi}{2} \; \mathrm{and}\; \gamma_0>\gamma:\nonumber\\
        & & \delta\rightarrow\alpha+\pi\nonumber\\
        & & \alpha\rightarrow\pi-\delta\nonumber\\
\end{eqnarray}
%
\begin{eqnarray}
        \mathrm{Where}&\;& \delta_0+d>\cfrac{\pi}{2} \; \mathrm{and}\; \gamma>2\pi-\gamma_0:\nonumber\\
        & & \delta\rightarrow\alpha-\pi\nonumber\\
        & & \alpha\rightarrow\pi-\delta\nonumber\\
\end{eqnarray}
%
\begin{eqnarray}
        \mathrm{Where}&\;& \delta_0-d<-\cfrac{\pi}{2} \; \mathrm{and}\; \gamma_0<\gamma<2\pi-\gamma_0:\nonumber\\
        & & \delta\rightarrow\alpha+\pi\nonumber\\
        & & \alpha\rightarrow-\pi-\delta\nonumber\\
\end{eqnarray}

An other way to draw circles in the sphere is to consider the point for which
we want to know celestial coordinates around a given angular distance as the
pole of a new coordinate system. In this system, points at given distance of
our central point are just points with $\pi/2-\delta$ and $\alpha$ running
between 0 and $2\pi$. We now can determine cartesian coordinates of
those points in this system and apply a rotation to go from the ``real'' system
and the system where the central point is the pole. In the new system we have:
%
\begin{eqnarray}
        X'&=&r\cos\alpha'\cos\delta' \nonumber\\
        Y'&=&-r\sin\alpha'\cos\delta' \nonumber\\
        Z'&=&r\sin\delta' \nonumber\\
\end{eqnarray}
%
Then the rotation matrix to go from the ``real'' system to the new is:
%
\begin{equation}
    R =
    \begin{pmatrix}
    \cos\pg\cfrac{\pi}{2}-\delta_0\pd\cos\alpha_0 & \sin\alpha_0 & \sin\pg\cfrac{\pi}{2}-\delta_0\pd\cos\alpha_0 \\
    -\cos\pg\cfrac{\pi}{2}-\delta_0\pd\sin\alpha_0 & \cos\alpha_0 & -\sin\pg\cfrac{\pi}{2}-\delta_0\pd\sin\alpha_0 \\
    -\sin\pg\cfrac{\pi}{2}-\delta_0\pd & 0 & \cos\pg\cfrac{\pi}{2}-\delta_0\pd \\
    \end{pmatrix}
\end{equation}
%
with $\vec{X}=R\vec{X'}$ where $\vec{X}=(X,Y,Z)$. Then we have just to convert
those coordinates in celestial angles using:
%
\begin{equation}
        \alpha=\left\{ \begin{array}{lcr}
                         -\mbox{arctan2}(Y,X)+2\pi & \mbox{if} & Y>0 \\
                         -\mbox{arctan2}(Y,X) & \mbox{else} & \\
                        \end{array}\right.\nonumber%
\end{equation}
%
\begin{equation}
        \delta=\mbox{sign}(Z)\arccos\left(\frac{\sqrt{X^2+Y^2}}{\sqrt{X^2+Y^2+Z^2}}\right)
\end{equation}
%
\begin{figure}[htb]
    \centering
    \includegraphics[width=0.6\linewidth]{figures/sdss/FracNonSpec.png}
    \caption{Fraction of galaxies non-spectroscoped in the SDSS versus the local density field computed in a
    1.5° radius region around galaxies not to close than this radius to the border of the survey. Density is in unit of galaxies
    per degree$^2$.}
\label{fig:fracnonspec}
\end{figure}
%
Fibre collisions are more probable in dense region of the sky in projection,
but for the mock catalogue we need to quantify this. In order to do that, we
have selected for all galaxies in the SDSS survey as defined previously
galaxies that are closer than 1.5° in angular size, which is the radius
of a plate used for spectroscopy in the SDSS\@. With that, we can estimate the
local density field $\Sigma_{1.5°}$ in unit of number of galaxies per
degree$^2$. In this selection, we can determine which galaxies had been
spectroscoped or not, and so we can estimate the fraction of non-spectroscoped
galaxies. We remove for computing it galaxies that are to close of the border
of the survey, and so galaxies which are closer than the radius selected can't
be used to search neighbours because some galaxies may be missed and can affect
our estimations. Results are shown in the figure (\ref{fig:fracnonspec}). We
can't see the trend we have expected with the density field, so we thought that
it can be due to the large region in which we consider galaxies and we ran the
same with a radius of 0.3°. Results are in figure
(\ref{fig:fracnonspec0.3}).
%
\begin{figure}[htb]
    \centering
    \includegraphics[width=0.6\linewidth]{figures/sdss/FracNonSpec03.png}
    \caption{Fraction of galaxies non-spectroscoped in the SDSS versus the
    local density field computed in a 0.3° radius region around galaxies not to
close than this radius to the border of the survey. Density is in unit of
galaxies per degree$^2$.}
\label{fig:fracnonspec0.3}
\end{figure}

In order to decide which redshift to assign to a galaxy in the mock catalogue
which has been chosen to have a photometric redshift, we have estimated the
distribution of photometric redshifts versus the spectroscopic redshift in the
SDSS sample of spectroscoped galaxies. Results show that a normal distribution
is a well fit for those distributions, so we get for this parameters in figure
(\ref{fig:evolnormred}).
%
\begin{figure}[htb]
    \centering
    \includegraphics[width=0.8\linewidth]{figures/sdss/EvolErrZphotSDSSDR8EqualBins}
    \caption{Parameters of a normal distribution for photometric redshifts
    versus spectroscopic redshifts in the SDSS.}
\label{fig:evolnormred}
\end{figure}

In the mock catalogue, we have interpolated this parameters and we assign a
photometric redshift for a galaxy chosen to be in fibre collision.
%
\section{Coverage of the SDSS}
%
For many computations in this thesis, we need to determine the surface covered
on the sky by the data in our selection. The way we have selected galaxies
allows us to easily determine if a point in the sky is in our area, so we can
use a Monte Carlo process to compute this area.

First, we generate a number $N$ of points around a point of coordinates
$(\alpha_0, \delta_0)$ with a maximal angular separation $\theta_{\max}$ which
is larger than the maximal angular separation in our sample. The fraction of
those points which reside in our selection area gives the area of the selection
in fraction of the area of points generation. This area is just
$2\pi\pg1-\cos\theta_{\max}\pd$. \comments{I think I don't have squared
angles in the expression of the solid angle.}. I have made this calculation for
different cone angle $\theta_{\max}$ and for different number of points
to see if we have a convergence in the value of the area. Results are shown on
figure (\ref{fig:sdss_area}).
\begin{figure}[htb]
    \centering
    \includegraphics[width=\linewidth]{figures/sdss/SDSS_area}
    \caption{Determination of the area of the SDSS for our selection with a
    Monte Carlo process. Results seem to converge on a value of 2.1993
steradians.}
\label{fig:sdss_area}\end{figure}
