\bartchapterimage{halo}
\bartthumb{thumb_halo}
\chapter{Halo mass functions}
%
A description of how to compute halo mass functions given simple models in some
articles.
%
\section{Theory}
%
\subsection{Definition}
%
By definition, the halo mass function by unit of comobile volume is the number
of halos with mass $M$ comprise between $M$ and $M+\dd{M}$. If $N$ is the
number of halos, the halo mass function $\phi\left({M}\right)$ can be written:
%
\begin{equation}
    \phi\left({M}\right)=\cfrac{\dd{N^2}}{\dd{M}\dd{V}}=\cfrac{\dd{n}}{\dd{M}}
\end{equation}
%
In this case, $n$ can be the comobile density of halos, or the CDF of the
density. In the latter case, we have:
%
\begin{equation}
    n\left({M,z}\right)=\int_0^M{\phi\left({M,z}\right)\dd{M}}
\end{equation}
%
and so:
%
\begin{equation}
    \frac{\dd{n}}{\dd{M}}=\frac{\dd}{\dd{M}}\int_0^M{\phi\left({M,z}\right)}\dd{M}=\frac{\dd}{\dd{M}}\left({\Phi\left({M,z}\right)}-{\Phi\left({0,z}\right)}\right)={\phi\left({M,z}\right)}
\end{equation}
%
where $\Phi$ is a primitive of $\phi$.
%
\subsection{In practice}
%
Cosmological simulations give results with $f\left(\sigma\right)$ a fitted function on
simulations. $\sigma\left({M}\right)$ is the variance in mass of the smoothed density
fields. We can link this function to the halo mass function by:
%
\begin{equation}
    \phi\left({M,z}\right)=\frac{\dd\ln{\sigma^{-1}}}{\dd{M}}\frac{\rho_m\left({z}\right)}{M}{f\left(\sigma\right)}=\frac{\rho_m\left({z}\right)}{M^2}\left|{{M}\frac{\dd\ln\sigma}{\dd{M}}}\right|{f\left(\sigma\right)}
\end{equation}
%
where the computation of $\sigma$ involves the power spectrum $P\left({k}\right)$ and
the filter for spectrum $\tilde{W}\left({k}\right)$:
%
\begin{equation}
    \sigma^2\left({M}\right) =
    \cfrac{1}{2\pi^2}\int_0^\infty{P\left({k}\right)
    \tilde{W}{\left({k}\right)}^2{k^2}\dd{k}}
\end{equation}
%
This form is time consuming for the computation of the halo mass function and
model dependent. In~\cite{vandenBosch+02}, there is a good approximation
for this formula which is resumed to:
%
\begin{equation}
    \sigma(M)=\sigma_8\frac{f(u)}{f(u_8)}
\end{equation}
%
with the function $f$:
%
\begin{equation}
    f(u)=64,087{(1+1,074{u^{0,3}}-1,581{u^{0,4}}+0.954{u^{0.5}}-0.185{u^{0.6}})}^{-10}
\end{equation}
%
and $u$, $u_8$ which are:
%
\begin{eqnarray}
    u&=&3.804e-4\Gamma{\left(\frac{Mh}{\Omega_{m,0}}\right)}^{1/3}\nonumber\\
    u_8&=&32\Gamma\nonumber\\
    \Gamma&=&\Omega_{m,0}h\exp\left[{-\Omega_b(1+\sqrt{2h}/\Omega_{m,0})}\right]\nonumber\\
\end{eqnarray}
%
Now, with this approximation, we can compute easily the derivative of $\sigma$
and:
\begin{equation}
    {\left({{M}\frac{\dd\ln\sigma}{\dd{M}}}\right)}^{-1}+\undemi =
    \frac{\left(-0.000310111 X^{1.7}+0.00225895 X^{1.6}-0.00505879 X^{1.5}-0.1 X^{1.2}\right)}{\left(-0.000328357 X^{1.8}+0.00310111 X^{1.7}-0.0090358 X^{1.6}+0.0101176 X^{1.5}\right)}
\end{equation}
%
with:
%
\comments{Check if we can used directly the power spectrum in the calculation
without too many CPU time consuming\ldots}
%
\begin{equation}
    X=\left(h \Omega _{m,0} e^{-\Omega _b \left(\frac{\sqrt{2} \sqrt{h}}{\Omega _{m,0}}+1\right)} \sqrt[3]{\frac{h M}{\Omega _{m,0}}}\right)
\end{equation}
