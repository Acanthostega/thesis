\bartchapterimage{formulas}
\chapter{Formulas}
\bartthumb{thumb_formulas}

\section{Introduction}

In this appendix are described the formulas used in all computations realized
during my thesis. Its just a simple way to share and verify that the job id
done correctly. References to those formulas are indicated too, in order to
improve search when some doubts are presents.\comments{Add a little more in the
introduction.}

\section{Formulas}

\subsection{Cosmology}

Formulas which are related to the cosmology.

\noindent\rule{\linewidth}{1pt}
%
The luminosity distance is defined as the relation between the galaxy flux $S$
and its absolute luminosity $L$ by:
%
\begin{equation}
	d_{\mathrm{lum}}=\sqrt{\cfrac{L}{4\pi{S}}}
\end{equation}
%
An analytical precise computations isn't possible, but numerical computations
exist. Although precise, numerical recipes aren't sufficiently fast in
practice. Some other analytical approximations of this distance was created.

For example, in \citet{WU+10}, an approximation good to 0.3\% is available for
a range of values in $\Omega_\Lambda$ compatible with WMAP and Planck
results.In this approximation we have:
%
\begin{equation}
	d_{\mathrm{lum}}\left({z}\right) = \cfrac{c}{3H_0}
    \cfrac{1+z}{\Omega_\Lambda^{1/6}
    {\left({1-\Omega_\Lambda}\right)}^{1/3}}\left[\Psi\left({x\left({0,\Omega_\Lambda}\right)}\right)-\Psi\left({x\left({z,\Omega_\Lambda}\right)}\right)\right]
\end{equation}
%
with:
%
\begin{equation}
	\Psi\left({x}\right)=3x^{1/3}{2^{2/3}}\left[1-\cfrac{x^2}{252}-\cfrac{x^4}{21060}\right]\\
\end{equation}
%
\begin{equation}
	x\left({\alpha}\right)=\ln\left({\alpha+\sqrt{\alpha^2+1}}\right)\\
\end{equation}
%
\begin{equation}
	\alpha\left({z,\Omega_\Lambda}\right) = 1+2\cfrac{\Omega_\Lambda}
    {1-\Omega_\Lambda}\cfrac{1}{{\left(1+z\right)}^3}
\end{equation}
%
The other distances are simply linked to this luminosity distance. The angular
distance $d_{\mathrm{ang}}$ and the proper distance $d_{\mathrm{pm}}$ are
$d_{\mathrm{lum}}(z)={(1+z)}^2d_{\mathrm{ang}}(z)=(1+z)d_{\mathrm{pm}}(z)$.

\noindent\rule{\linewidth}{1pt}
%
The element of comoving volume is expressed using the Robertson-Walker metric
as:
%
\begin{equation}
	\dd{V}=\cfrac{c}{H\left({z}\right)}{d_{\mathrm{pm}}\left({z}\right)}^2\dd{\Omega}\dd{z}
\end{equation}

\noindent\rule{\linewidth}{1pt}
%
The evolution of the fraction of matter, and dark energy is the following:
%
\begin{equation}
	\Omega_m\left({z}\right)=\Omega_{m,0}
    \cfrac{{\left(1+z\right)}^3}{E{\left({z}\right)}^2}
\end{equation}
%
\begin{equation}
	\Omega_\Lambda\left({z}\right) =
    \cfrac{\Omega_{\Lambda,0}}{E{\left({z}\right)}^2}
\end{equation}
%
where $z$ is the redshift and the subscript 0 refers to the actual value of the
parameter.

\noindent\rule{\linewidth}{1pt}
%
The distance modulus represents the magnitude difference betweenthe observed
flux of the galaxy and it would be if the galaxy was at a distance of 10$pc$.
So it's:
%
\begin{equation}
	DM\left({z}\right)=5\log_{10}\left(\cfrac{d_{\mathrm{lum}}\left({z}\right)}{10pc}\right)%
\end{equation}
%
where $z$ is the redshift of the galaxy and $d_{\mathrm{lum}}$ is the
luminosity distance.

\noindent\rule{\linewidth}{1pt}
%
The apparent magnitude $m$ of galaxy in the perfect case where isn't
K-correction, extinction\ldots, is just:
%
\begin{equation}\label{eq:magappdm}
	m=M+DM\left({z}\right)%
\end{equation}
%
where $M$ is the absolute magnitude of this galaxy in the same band of $m$ and
$DM(z)$ is the distance modulus at redshift $z$.

\noindent\rule{\linewidth}{1pt}
%
Magnitudes are defined at a given constant which is the same for each object
so:
%
\begin{equation}
	M-M_\odot=-2.5\log_{10}\left(\cfrac{L}{L_\odot}\right)%
\end{equation}
%
where $M$ is absolute magnitude, $L$ the luminosity of the object and $\odot$
refers to Sun's quantities. We can determined the luminosity by this relation
which gives:
%
\begin{equation}
	\cfrac{L}{L_\odot}=10^{0.4\left({M_\odot-M}\right)}
\end{equation}

\noindent\rule{\linewidth}{1pt}
%
For galaxies at a given redshift $z$, we can see all galaxies with an absolute
magnitude lower than (using equation (\ref{eq:magappdm})):
%
\begin{equation}
    m_{\lim}=M+DM\left({z}\right)%
\end{equation}
%
where $m_{\mathrm{\lim}}$ is the apparent magnitude limit for a survey, and $M$
is the absolute magnitude threshold to be seen at this redshift.

\noindent\rule{\linewidth}{1pt}
%
The virial radius $r_\Delta$ is defined as the radius at which the density is
$\Delta$ times the critical density of the Universe. So we have:
%
\begin{equation}\label{eq:radcrit}
	\rho\left({r_\Delta}\right)=\Delta\rho_c
\end{equation}
%
with $\rho_c=\cfrac{3H{\left({z}\right)}^2}{8\pi{G}}$.

If we suppose that the density is constant in this radius, we have:
%
\begin{equation}
	\Delta\cfrac{3H{\left({z}\right)}^2}{8\pi{G}} =
    \cfrac{M_\Delta}{4\pi{r_\Delta}^3/3}
\end{equation}
%
where $M_\Delta$ is the virial mass. We can now defined three quantities, the
virial mass as:
%
\begin{equation}
	M_\Delta=\cfrac{\Delta{H{\left({z}\right)}^2{r_\Delta}^3}}{2G}
\end{equation}
%
the virial radius as:
%
\begin{equation}
    r_\Delta={\left(\cfrac{2 G M_\Delta}
    {\Delta{H{\left({z}\right)}^2}}\right)}^{1/3}
\end{equation}
%
and the virial velocity as:
%
\begin{equation}
	v_\Delta=\sqrt{\cfrac{G M_\Delta}{r_\Delta}}=\sqrt{\cfrac{\Delta}{2}} H\left({z}\right){r_\Delta}
\end{equation}

\noindent\rule{\linewidth}{1pt}
%
Sometimes, the density at the virial radius isn't defined in relation with the
critical density but instead with mean density of the Universe. So the equation
(\ref{eq:radcrit}) becomes:
%
\begin{equation}
	\rho\left({r_\Delta}\right)=\Delta\rho_m=\Delta{\Omega_m}\rho_c
\end{equation}
%
We can treat this situation in the same way as previously, but formally with
$\Delta\rightarrow\Delta\Omega_m$.

\noindent\rule{\linewidth}{1pt}
