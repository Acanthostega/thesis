\bartchapterimage{opo0328a.jpg}
\chapter{$q$-Gaussians (or Tsallis) distributions}
\label{cha:q_gaussians_or_tsallis_distributions}
\bartthumb{opo0328a.png}
\minitoc%

First computations of the probability to belong in a group for MAGGIE where
not satisfying when compared to the probability estimated directly from the
cosmological simulation. Probability contours let us think in a stronger
truncation in the velocity distribution for high velocities in the
simulation than that of the assumed Gaussian distribution, creating
discrepancies between our expectations and data. A natural distribution with
this property is the $q$-Gaussian or Tsallis distribution, with an
additional free parameter, which one adjusted, allows to get a stronger
truncation in the velocity distribution. But the resulting distribution
function of the particle system becomes quite complex and the analytical
computations difficult in some cases, resumed in the following sections.

\section{$q$-Gaussian (or Tsallis) distributions}
\label{sec:q_gaussian_or_tsallis_distributions}

\citet{Tsallis+88} developed a generalization of the entropy, non extensive,
in terms of an additional parameter $q$ leading to systems that are not at
the equilibrium. According to \citet{Hansen+06}, radial and tangential
velocity distributions of dark matter particles in halos in cosmological
simulations doesn't exactly follow a Gaussian distribution, but are more
accurately adjusted by a Tsallis distribution. A possible explanation is
that halos are not fully at the equilibrium at our epoch.

Radial $v_r$ and tangential $v_t$ velocity distributions at a given point of
coordinates $\textbf{r}$ are of the following forms:
%
\begin{equation}
    \frac{\dd N}{\dd v_r} =
    A_r{\left(1-B_r{\left(\frac{v_r}{\sigma_r}\right)} ^2\right)}^{\alpha_r}
\end{equation}
%
\begin{equation}
    \frac{\dd N}{\dd v_t}=
    A_t v_t
    {\left(1-B_t{\left(\frac{v_t}{\sigma_t}\right)}^2\right)}^{\alpha_t}
\end{equation}
%
In the two above equations, integrating over the corresponding velocities
gives us the number of particles at a given coordinates by unit of volumes,
in other words the density profile $\nu\left(\textbf{r}\right)$. Integrating
over the second moment, we have the velocity dispersion, an other constraint
to determine the normalization factor.
%
\begin{equation}
    \label{eq:mom0}
    \int f \left(\textbf{r},\textbf{v}\right)\dd\textbf{v}=
    \int\frac{\dd N}{\dd v_i}\dd v_i=
    \nu\left(\textbf{r}\right)
\end{equation}
%
\begin{equation}
    \int{v_i}^2 f \left(\textbf{r},\textbf{v}\right)\dd\textbf{v}=
    \int{v_i}^2\frac{\dd N}{\dd v_i}\dd v_i=
    {\sigma_i}^2\nu\left(\textbf{r}\right)
\end{equation}
%
In the case of the tangential velocity $v_t^2=v_\theta^2+v_\phi^2$, the
dispersion is simply $\sigma_t^2=\sigma_\theta^2+\sigma_\phi^2$. In the
following sections, we will assume isotropy and so
$\sigma_\theta=\sigma_\phi$.

For the integration over $v_r$, we need to take into account the two cases
$B_r>0$ and $B_r<0$.
%
\begin{equation}
    \int_{-\infty}^\infty{\left(1+x^2\right)}^\alpha\dd x=
    \frac{\sqrt\pi\Gamma\left(-\undemi-\alpha\right)}
    {\Gamma\left(-\alpha\right)}\hspace{0.5cm}\alpha<-\undemi
\end{equation}
%
\begin{equation}
    \label{eq:ref1}
    \int_{-1}^1{\left(1-x^2\right)}^\alpha\dd x=
    \frac{\sqrt\pi\Gamma\left(1+\alpha\right)}
    {\Gamma\left(\frac{3}{2}+\alpha\right)}\hspace{0.5cm}\alpha>-1
\end{equation}
%
\bartrefequation{mom0} gives us normalizations by substitution to have the
equations \bartrefequation{ref1}:
%
\begin{equation}
    \frac{A_r\sqrt\pi\sigma_r\Gamma\left(-\undemi-\alpha_r\right)}
    {\sqrt{-B_r}\Gamma\left(-\alpha_r\right)}=
    \nu\left(\textbf{r}\right)\hspace{0.5cm}\alpha_r<-\undemi
\end{equation}
%
\begin{equation}
    \frac{A_r\sqrt\pi\sigma_r\Gamma\left(1+\alpha_r\right)}
    {\sqrt{B_r}\Gamma\left(\frac{3}{2}+\alpha_r\right)}=
    \nu\left(\textbf{r}\right)\hspace{0.5cm}\alpha_r>-1
\end{equation}
%
For second moment equations, we use the following equations:
%
\begin{equation}
    \int_{-\infty}^\infty x^2{\left(1+x^2\right)}^\alpha\dd x=
    \frac{\sqrt\pi\Gamma\left(-\frac{3}{2}-\alpha\right)}
    {2\Gamma\left(-\alpha\right)}\hspace{0.5cm}\alpha<-\frac{3}{2}
\end{equation}
%
\begin{equation}
    \label{eq:ref2}
    \int_{-1}^1 x^2{\left(1-x^2\right)}^\alpha\dd x=
    \frac{\sqrt\pi\Gamma\left(1+\alpha\right)}
    {\Gamma\left(\frac{5}{2}+\alpha\right)}\hspace{0.5cm}\alpha>-1
\end{equation}
%
giving us:
%
\begin{equation}
    \frac{A_r\sqrt\pi{\sigma_r}^3\Gamma\left(-\frac{3}{2}-\alpha\right)}
    {2{\left(-B_r\right)}^{3/2}\Gamma\left(-\alpha\right)}=
    {\sigma_r}^2\nu\left(\textbf{r}\right)\hspace{0.5cm}\alpha_r<-\frac{3}{2}
\end{equation}
%
\begin{equation}
    \frac{A_r\sqrt\pi{\sigma_r}^3\Gamma\left(1+\alpha_r\right)}
    {2{B_r}^{3/2}\Gamma\left(\frac{5}{2}+\alpha_r\right)}=
    {\sigma_r}^2\nu\left(\textbf{r}\right)\hspace{0.5cm}\alpha_r>-1
\end{equation}
%
Finally, normalizations are:
%
\begin{equation}
    B_r=\frac{1}{3+2\alpha_r}
\end{equation}
%
\begin{equation}
    A_r=\frac{\nu\left(\textbf{r}\right)\Gamma\left(-\alpha_r\right)}
    {\Gamma\left(-\undemi-\alpha_r\right)}
    \frac{1}{\sqrt{-\left(3+2\alpha_r\right)\pi}\sigma_r}
    \hspace{0.5cm}\alpha_r<-\frac{3}{2}
\end{equation}
%
\begin{equation}
    A_r=
    \frac{\nu\left(\textbf{r}\right)\Gamma\left(\frac{3}{2}+\alpha_r\right)}
    {\Gamma\left(1+\alpha_r\right)}
    \frac{1}{\sqrt{\left(3+2\alpha_r\right)\pi}\sigma_r}
    \hspace{0.5cm}\alpha_r>-1
\end{equation}

Integrating other tangential velocities, we use following results:
%
\begin{equation}
    \int_0^\infty x{\left(1+x^2\right)}^\alpha\dd x=
    -\frac{1}{2\left(1+\alpha\right)}\hspace{0.5cm}\alpha<-1
\end{equation}
%
\begin{equation}
    \int_0^1 x{\left(1-x^2\right)}^\alpha\dd x=
    \frac{1}{2\left(1+\alpha\right)}\hspace{0.5cm}\alpha>-1
\end{equation}
%
\begin{equation}
    \int_0^{1\,\mbox{or}\,\infty} x^3{\left(1\pm x^2\right)}^\alpha\dd x=
    \frac{1}{2\left(1+\alpha\right)\left(2+\alpha\right)}
    \hspace{0.5cm}\alpha>-1\,\mbox{or}\,\alpha<-2
\end{equation}
%
giving:
%
\begin{equation}
    B_t=\frac{1}{\left(2+\alpha_t\right)}
\end{equation}
%
\begin{equation}
    A_t=\frac{\nu\left(\textbf{r}\right)\left(1+\alpha_t\right)}
    {\sigma_t\left(2+\alpha_t\right)}
\end{equation}

\section{Choice of a distribution function}
\label{sec:choice_of_a_distribution_function}

The global velocity distribution is the combination of the radial and
tangential distributions, not as easy as wanted.

\subsection{Similar to Gaussian case}
\label{sub:similar_to_gaussian_case}

We choose a form similar ot the Gaussian case, but not identical in the
sense of a strict product of independent variables. We have:
%
\begin{equation}
    f\left(\textbf{r},\textbf{v}\right)=
    A\left(\alpha\right)
    {\left(1-B\left(\alpha\right)\left(
                {\left(\frac{v_r}{\sigma_r}\right)}^2+
                {\left(\frac{v_{\theta}}{\sigma_{\theta}}\right)}^2+
                {\left(\frac{v_{\phi}}{\sigma_{\phi}}\right)}^2
    \right)\right)}^\alpha
\end{equation}
%

We find normalizations in the same way as previously. We can doubt in the
choice of the limit velocity for the integration in the case where
$B\left(\alpha\right)>0$. But if we choose to impose a limit on one
velocity, the constraints on the two others becomes the consideration of the
three velocities as a single one with the good substitution in the
integration. We just to consider the triplet of velocities following the
constraint ${\left(\frac{v_r}{\sigma_r}\right)}^2+
{\left(\frac{v_\theta}{\sigma_\theta}\right)}^2+
{\left(\frac{v_\phi}{\sigma_\phi}\right)}^2<1$. In consequence, we find for
the normalizations:
%
\begin{equation}
    A\left(\alpha\right)=
    \frac{\nu\left(\textbf{r}\right)}
    {{\left|5/2+\alpha\right|}^{3/2}{\left(2\pi\right)}^{3/2}
        \sigma_r{\sigma_\theta}^2}
    \frac{\Gamma\left(-\alpha\right)}
        {\Gamma\left(-3/2-\alpha\right)}\hspace{0.5cm}\alpha<-\frac{5}{2}
\end{equation}
%
\begin{equation}
    A\left(\alpha\right)=
    \frac{\nu\left(\textbf{r}\right)}
    {{\left|5/2+\alpha\right|}^{3/2}{\left(2\pi\right)}^{3/2}
    \sigma_r{\sigma_\theta}^2}
    \frac{\Gamma\left(5/2+\alpha\right)}{\Gamma\left(1+\alpha\right)}
    \hspace{0.5cm}\alpha>-1
\end{equation}
%
\begin{equation}
    B\left(\alpha\right)=\frac{1}{5+2\alpha}
\end{equation}
%
Integrating over $v_t$ we get the radial distribution and the tangential
distribution if integrating over $v_r$. The definitions of the $q$-Gaussian
distributions imply $\alpha_r=1+\alpha$ et $\alpha_t=1/2+\alpha$. But
fitting these distributions to the data of the cosmological simulation gives
different values of $\alpha$ for the radial and tangential distribution,
hence the model of the global velocity distribution is not adapted and we
need to find an other expression.

\remark{%
    However, if we want to obtain the velocity distribution along the
    line-of-sight we need to integrate this form on the two velocities
    perpendicular to the line-of-sight. Computations are simple if the
    quadratic form is transformed into a canonical one, useful with the
    previous integral definitions.
}

\subsection{Different $\alpha$ forms}
\label{sub:different_alpha_forms}

We treat the case:
%
\begin{equation}
    f\left(\textbf{r},\textbf{v}\right)=
    A{\left(1-B_r{\left(\frac{v_r}{\sigma_r}\right)}^2\right)}^{\alpha_r}
    v_t{\left(1-B_t{\left(\frac{v_t}{\sigma_t}\right)}^2\right)}^{\alpha_t}
\end{equation}
%
Constraints give:
%
\begin{equation}
    B_t=
    \frac{1}{2\left(2+\alpha_t\right)}
\end{equation}
%
\begin{equation}
    B_r=
    \frac{1}{\left(3+2\alpha_r\right)}
\end{equation}
%
and following cases, we obtain for $A$ in the \bartreftable{tsallis} the
different normalizations.
%
\begin{table}[htb]
    \centering
    \caption{Table of coefficient for the normalization in different
        cases.\label{tab:tsallis}}
    \resizebox{\columnwidth}{!}{%
	\begin{tabular}{ccc}
        \toprule
        & $\alpha_r<-3/2$ & $\alpha_r>-1$ \\
        \midrule%
        & & \\
		$\alpha_t$ $<$ $-2$ &
        $\cfrac{\left(1+\alpha_t\right)\nu\left(\textbf{r}\right)
            \Gamma\left(-\alpha_r\right)}{\sqrt{-\pi\left(3+2\alpha_r\right)}
            \sigma_r{\sigma_t}^2\left(2+\alpha_t\right)
            \Gamma\left(-1/2-\alpha_r\right)}$ &
        $\cfrac{\left(1+\alpha_t\right)\nu\left(\textbf{r}\right)
            \Gamma{\left(3/2+\alpha_r\right)}^{3/2}}
            {\sqrt{-\pi\left(3+2\alpha_r\right)}\sigma_r{\sigma_t}^2
            \left(2+\alpha_t\right)\Gamma\left(1+\alpha_r\right)
            \Gamma{\left(5/2+\alpha_r\right)}^{1/2}}$ \\
        & & \\
        \midrule%
        & & \\
		$\alpha_t$ $>$ $-2$ &
            $\cfrac{\left(1+\alpha_t\right)\nu\left(\textbf{r}\right)
            \Gamma\left(-\alpha_r\right)}{\sqrt{-\pi\left(3+2\alpha_r\right)}
            \sigma_r{\sigma_t}^2\left(2+\alpha_t\right)
            \Gamma\left(-1/2-\alpha_r\right)}$ &
        $\cfrac{\left(1+\alpha_t\right)\nu\left(\textbf{r}\right)
            \Gamma{\left(3/2+\alpha_r\right)}^{3/2}}
            {\sqrt{-\pi\left(3+2\alpha_r\right)}\sigma_r{\sigma_t}^2
            \left(2+\alpha_t\right)\Gamma\left(1+\alpha_r\right)
            \Gamma{\left(5/2+\alpha_r\right)}^{1/2}}$ \\
        & & \\
        \bottomrule
	\end{tabular}
    }
\end{table}

Integrating to obtain the radial and tangential distribution, we see that we
have two different values of $\alpha$, each one equal to the tangential and
radial $\alpha$. This seems to correspond with what observed in the
cosmological simulation. The problem is that to obtain the line-of-sight
velocity distribution, the equation doesn't have an analytical expression,
and in consequence not useful in the computation of the probability of
MAGGIE\@. So we choose to abandon this model.

\section{Generate $q$-Gaussian distributions}
\label{sec:generate_q_gaussian_distributions}

\subsection{One dimension}
\label{sub:one_dimension}

The distribution function is expressed as:
%
\begin{equation}
    f\left(\textbf{r},\textbf{v}\right)=
    A{\left(
        1-B\left(\frac{{\left(v-\mu\right)}^2}{\sigma^2}\right)
    \right)}^{\frac{q}{1-q}}
\end{equation}
%
where we replaced $\alpha$ with its equivalent $q$ which is
$q/\left(1-q\right)$, making the modeling easier since there is no cut in
values of the $q$ parameter. The distribution is centered in $\mu$ with
dispersion $\sigma$. According to \citet{Thistleton+06}, random numbers
following the $q$-Gaussian distribution are expressed as:
%
\begin{equation}
    Z_1=\sqrt{-2\left(\cfrac{3-q}{1+q}\right)\ln_{\frac{3q-1}{q+1}}U_1}
    \cos\left(2\pi U_2\right)
\end{equation}
%
\begin{equation}
    Z_2=\sqrt{-2\left(\cfrac{3-q}{1+q}\right)\ln_{\frac{3q-1}{q+1}}U_1}
    \sin\left(2\pi U_2\right)
\end{equation}
%
with:
%
\begin{equation}
    \ln_q x=\frac{x^{1-q}-1}{1-q}
\end{equation}
%
where $U_1$ and $U_2$ are two random variables following a uniform
distribution, with their values between 0 and 1. To generate a one
dimensional Tsallis with dispersion $\sigma$ and mean $\mu$, the following
random variables are sufficient:
%
\begin{equation}
    Z = \sigma Z_i + \mu
\end{equation}
%
where $i\in \{1,2 \}$.

\subsection{Two dimensions case}
\label{sub:two_dimensions_case}

In the case of the tangential distribution:
%
\begin{equation}
    f\left(\textbf{r},\textbf{v}\right)=A v
    {\left(
        1-B\left(\frac{{\left({v-\mu}\right)}^2}{\sigma^2}\right)
    \right)}^{\frac{q}{1-q}}
\end{equation}
%
We force $\mu=0$ for an easier computation. The cumulative distribution
function is:
%
\begin{equation}
    F_X\left(x\right)=\int_0^x A v
    {\left(1-B{\left(\frac{v}{\sigma}\right)}^2\right)}^{\frac{q}{1-q}}\dd v
\end{equation}
%
If $U$ is an uniform random variable between 0 and 1, hence
$U=F_X\left({X}\right)$. Inverting the relation, we find $X$ following the
distribution. In all cases where $q<1$ and $1<q<2$, we can find:
%
\begin{equation}
    X=\sigma\sqrt{2\left(\frac{2-q}{1-q}\right)\left(1-U^{1-q}\right)}
\end{equation}
%
With the mean of the distribution:
%
\begin{equation}
    X=\mu+\sigma\sqrt{2\left(\frac{2-q}{1-q}\right)\left(1-U^{1-q}\right)}
\end{equation}

\section{Cumulative distribution functions}
\label{sec:cumulative_distribution_functions}

The cumulative distribution function can be useful in the situation where we
search parameters fitting an unknown distribution by the Kolmogorov-Smirnov
method for example.

\subsection{One dimension case}
\label{sub:one_dimension_case}

By definition of the cumulative distribution function:
%
\begin{equation}
    F_X\left(x\right)=\int_{-\infty}^x f\left(x\right)\dd x
\end{equation}
%
with $f\left(x\right)$ the distribution function. With the dispersion
$\sigma$ and the bias $\mu$:
%
\begin{equation}
    F_X\left(x\right)=
    \frac{A_r\sigma}{\sqrt{\left|B_r\right|}}
    \left(
        \frac{\sqrt\pi}{2} f^1\left(\alpha\right)+
        \frac{\sqrt{\left|B_r\right|}}{\sigma}\left(x-\mu\right)
        \HGG{\undemi}{-\alpha}{\frac{3}{2}}
        {-B_r{\left(\frac{x-\mu}{\sigma}\right)}^2}\right)
\end{equation}
%
where $A_r$ and $B_r$ are the same coefficients found previously,
$\mathcal{H}{2{\rm{F}}1}$ the hyper-geometric function 2F1 and $f^1$ a
function such:
%
\begin{eqnarray}
    f^1\left(\alpha\right)&=&
        \frac{\Gamma\left(-\undemi-\alpha\right)}{\Gamma\left(-\alpha\right)}
        \;\;\;\alpha<-\undemi\\\nonumber
    &=&\frac{\Gamma\left(1+\alpha\right)}{\Gamma\left(-\alpha\right)}
        \;\;\;\alpha>-1\\
\end{eqnarray}

\subsection{Two dimensions case}
\label{sub:cdf_two_dimensions_case}

In this case the computation is easier:
%
\begin{equation}
    F_X\left(x\right)=
        \frac{A_t\sigma^2}{B_t}
        \frac{1}{2\left(1+\alpha\right)}
            \left(1-{\left(
                1-B_t{\left(\frac{x-\mu}{\sigma}\right)}^2
            \right)}^{1+\alpha}\right)
\end{equation}
%
with $A_t$ and $B_t$ identical to the coefficients determined previously.
