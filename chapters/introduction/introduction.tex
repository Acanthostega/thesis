\chapter{Introduction}
%
Since the discovery of galaxies as distant objects from the Milky Way
\citep{Hubble+29}, many work has been done to understand how they formed and
what drives their observable properties (morphologies, color\ldots) at our
epoch and earlier in their evolution \citep{Benson+10,Silk+12,Silk+13}. The
combination of the structure formation of cold dark matter (CDM) particles
and their history \citep{Zentner+07}, with the baryon physics inside dark
matter halos \citep{Kravtsov+12} has been quite successful in reproducing
and explaining the observations from galaxy surveys. But there are still
some lacks in the galaxy formation scenario, which are headaches to solve
for theorists \citep{Weinmann+12}. A frequent solution to resolve this
puzzle is to introduce different recipes in galaxy formation simulations to
account for the missing physics that can reduce these discrepancies.
%
Such an example (and the most known problem of $\rm \Lambda$CDM) is the
overabundance of dwarf galaxies predicted by semi-analytical models (SAM) in
simulation of galaxy formation. Reducing their number implies to eject the
excessive baryons through several physical processes (feedback,
\citet{Brooks+13}) in order to make the dwarfs not resolvable. Such typical
processes are, for example, supernovae winds \citep{Hirschmann+13} or ram
pressure stripping. But introducing them leads to a more and more
complex scenario, and doesn't allow to clearly distinguish the effect of
each physical process on the galaxy evolution.

The galaxy formation is tightly correlated to the galaxy environment.
Indeed, galaxies are gregarious, leaving in different hosts environments
from isolated galaxies, to pairs, groups and clusters. This environment
impacts on galaxy properties in different manner, at different epochs,
through a large range of possible physical processes. But not all them are
important according to the redshift and environment of galaxies. The
characterization of the major physical process at work inside environments
should improve the predictions of SAM, by including more precise models and
recipes in the code, directly extracted from the analysis of the
observations. Moreover, this should also improve the galaxy formation
scenario constructed until now, and work as a test for this scenario. This
goal can only be achieved by an optimal definition of the galaxy
environment, in other words, an optimal selection of galaxy group and
clusters.

\section{The importance of galaxy groups}
\label{sec:the_importance_of_galaxy_groups}

\subsection{Galaxy group physics}
\label{sub:galaxy_group_physics}

Observed galaxy groups are a direct consequence of this hierarchical growth
of structure. Galaxies therein are affected by this growth since they formed
in dark matter sub-halos that merged with most massive halos along the
Universe expansion according to the hierarchical scenario. So their
properties must be correlated with their parent dark matter halo and reflect
their physical processes history inside it. Some evidence of such a
modulation of galaxy properties were already observed previously on the
galaxy luminosity \citep{Robotham+10} and stellar mass \citep{Yang+09}
functions, with the galaxy environment.

Galaxies can be classified in two distinct populations with their
properties: a blue population of gas rich and young stellar population and a
red one, poor in gas with an old stellar population. This bi-modality is
also visible in their morphologies where red galaxies are essentially
ellipsoidal and blue galaxies are spiral. A segregation of these galaxies
exists with the environment close to our epoch (low redshifts): red galaxies
lie in dense environment such as clusters, while the blue population is more
present in the field (outside dense environments as clusters or groups).
This is clearly an effect of the environment, where in clusters, the dense
region allows the intra-cluster gas to be hot enough to stop the star
formation of galaxies, leading to an old and red stellar population. In
lower dense region, mergers and various interactions implying galaxies are
frequent and boost the star formation.

But some other properties lead to discrepant results. For example, the
specific star formation rate (SSFR) doesn't show a dependence on the
environment according to~\cite{Peng+10} for high stellar mass galaxies, but
following \citet{vonderLinden+10}, there is clearly a trend of decline of
the SSFR for star forming galaxies towards groups center (for all galaxy
masses). The results of~\cite{Peng+10} are surprising since in groups and
clusters, galaxies are massive (in stars) due to the cannibalism that
contributed in the past to the formation of the structure. Moreover, the
dense environment should quench the star formation in galaxies due to the
intra-cluster gas preventing the formation of cold clouds, and so varying
with the distance to the center of the group. This contradiction is possibly
explained by the selection of a tracer for the environment in~\cite{Peng+10}
that doesn't distinguish between the two kind of environment: the local one
related to the position of the galaxy relatively to its halo, and the global
environment that characterizes the total mass embedded in the parent halo of
the galaxy. Moreover, a clean characterization of the environment from the
redshift space is difficult since the redshift distortions
\citep{Jackson+72}, called also Fingers-of-God \citep{Tully+78}, caused by
the velocity dispersion of the galaxy group can create overlapping between
galaxies of foreground or background groups.

\section{Galaxy formation}
\label{sec:galaxy_formation}

Several physical processes are at work inside galaxy groups because of the
galaxy over-density relatively to the background field. They are mainly
caused by interactions between each galaxy and/or the group. Galaxy mergers
(essentially major mergers involving two galaxies of equivalent masses) are
expected to morphologically transform galaxies to spheroidal
\citep{Naab+99,Bournaud+05}, and to create star formation bursts inside
merging galaxies \citep{Cox+08,Teyssier+10}. In the other hand, the dense
environment acts too on galaxy properties. Tidal forces exerted by the group
and the ram pressure stripping can remove the outer gaseous regions in
orbiting galaxies leading to a quenching of the star formation
\citep{Larson+80,Bekki+13}.

Some of these intra-group physics were already, more or less well,
introduced in SAM of galaxy formation
\citep{Okamoto+03,Lanzoni+05,Font+08,Guo+11}. But all these methods tend to
over-simplify, by use of simple formulas, very complex processes depending
on several parameters and the galaxy environment. A better modeling of the
physics involved in galaxy group should improve the SAM and correct their
difficulties in fully describing the observed Universe. This can only be
achieved with a good definition of the environment for galaxies, and galaxy
groups in redshift space are exactly this definition of environment.

\subsection{Galaxy groups as tests}
\label{sub:galaxy_groups_as_tests}

Galaxy groups are not just limited to test and improve the models for the
galaxy formation theory, but also appear in other astrophysical domains. In
cosmology, they are a tool to access the cosmological parameters, such as
the dark energy fraction \citep{Wang+98}. General relativity can be tested
with them \citep{Wojtak+11}. \com{and so many other examples\ldots}

\section{Characterizing environment}
\label{sec:characterizing_environment}

\subsection{History}
\label{sub:history}

Many galaxy group catalogs were already published, usually following the
first publications of data from galaxy surveys. First attempts were done
with human selections \citep{Abell+58,Zwicky+61,Rose+76}. The selection was
done based on non physical assumptions on galaxy groups, with certain
criteria for a visual over-density of galaxies.

Then the percolation or Friends-of-Friends (FoF) algorithm followed, based
on the knowledge of galaxy physics at this epoch
\citep{Huchra+82,Nolthenius+87}. One of its advantages is that it is based
on a physical choice for the way to link galaxies between them in groups. A
linking length is used to relate galaxies that are closer than this distance
in redshift space. This needs the use of two different linking lengths in
the line-of-sight and perpendicular directions to avoid the redshift
distortions effect. But as argued in \citet{DM+14a}, with some priors in the
galaxy distribution, these links must be adjusted to the mass of the group
(i.e.\ its richness) to be complete in the galaxy selection. More recently,
\citet{Eke+04} and \citet{Berlind+06} published too galaxy groups catalogs
from the application of the percolation algorithm, but taking into account,
in their selection, the incompleteness induced by the galaxy surveys used.

\citet{Marinoni+02} developed a method similar to FoF but with the use of a
redshift space partitioned into Voronoi cells, to have an initial seed for
the over-density (Voronoi cells volume trace the galaxy density) around each
galaxy. But this method suffers from the necessity to use it in small
surveys in angle because of the difficulty to create a tessellation of the
celestial sphere directly.

With the increasing advances in the galaxy formation processes, capacities
of numerical computation and predictions of the cosmological simulations,
started to appear Bayesian algorithms that used priors on galaxy groups to
improve their extraction from galaxy surveys. \citet{Yang+05,Yang+07}
developed an iterative method to select galaxy groups based on a density
contrast criterion, which uses assumptions based on cosmological simulation
results for the density profile of groups.

Galaxy surveys have limitations that are difficult to overcome in galaxy
group algorithms. In the case of surveys of photometric redshifts,
probabilistic Friends-of-Friends were developed to attempt avoiding the
large (and sometimes catastrophic) uncertainties in redshift measures
\citep{Liu+08}. Then, probability was used to improve the membership of
galaxies inside their groups, as \citet{DominguezRomero+12}, allowing a more
flexible way to affect galaxies.

Finally, group finding algorithms continue their insertion of galaxy
formation results, combining it with the advantage of geometrical methods.
An example is \citet{MunozCuartas+12} that used a FoF applied on dark matter
halos associated to galaxies, with the initial assumption that all galaxies
are their own halo, and so the central galaxy mass is a tracer of the
density field (the most massive central galaxies are associated to the most
massive halos).

\subsection{And now\ldots?}
\label{sub:and_now}

Actual and next generations of galaxy surveys allow us to probe galaxy
groups in different aspects, each of them with their improvements and
limits. Sloan Digital Sky Survey (SDSS), with its around one million of
spectroscoped galaxies, gives us a good overview of the density field for a
large range of redshifts. But this abundance of precise redshifts as the
counterpart that not all galaxies have spectroscopic redshifts, and around
5--10\% of galaxies, because of the fiber collision problem
\citep{Blanton+03}, need to fall back to photometric redshifts, more
inaccurate. The Galaxy And Mass Assembly, at its final stage, will contain
around $300\;000$ galaxies with a spectroscopic redshift \citep{Hopkins+13},
less than the SDSS\@. But the completeness of the sample will be higher than
the SDSS with $\simeq99\%$ of the sample spectroscoped. The counterpart is a
less precise measurement of galaxies recession velocities
\citep{Robotham+11,Hopkins+13}. Moreover, the adjoining angular size is
lower because of the fragmentation of the survey regions. But those galaxy
samples are from different sky region allowing to take into account the
cosmic variance in the statistics.
%
In consequence, galaxy group algorithms must be flexible to be applied and
give the same result in many, different and (surprisingly) creative future
galaxy survey projects. Their common limitations and advantages must be take
into account when developing it.

Galaxy group algorithms give different results, maybe caused by
a lack of interlopers removal and/or bad completeness. Difficult to estimate
since tested in differently constructed mock catalogs. \com{Need to really
check that!}

So we need to go beyond the usual standard and static definition of groups
and work with the inevitable polluted environment of extracted galaxy
groups. So here is MAGGIE, a probabilistic galaxy group algorithm that
avoids the importance of interlopers in the galaxy group properties
observed.
