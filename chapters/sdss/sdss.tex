\bartchapterimage{heic0506a.jpg}
\chapter{SDSS-DR10 analysis}
\label{cha:sdss}
\bartthumb{heic0506a.png}
\minitoc%

\section{Introduction}

An application of MAGGIE, our optimal probabilistic galaxy group algorithm, on
a real galaxy survey implies an analysis of the galaxy sample. We must
understand the various incompletenesses it suffers in order to be able to
correct them. Here we show an analysis we performed on the Sloan Digital Sky
Survey, with the various problems we encountered.

\section{Analysis}

\subsection{Definitions}

Strips are bands of observations along great circles of the survey. Each of
them is composed of six parallel scanlines (of 13 arcmin wide) with gaps of
approximately the same width between them. Two strips make a single stripe
of 2.5°. Each scanline include all the data (in $ugriz$), and is divided in
fields (that can overlap). So when accessing to an observation at a given
position in the sky, we access a specific field. A given observation is
completely defined by its run number, the number of the camcol of the
scanline and by the field number.

On each field, the pipeline of the SDSS is applied for the objects
extraction. They are detected as pixels over-densities relatively to the
background. With this method, multiple real and different objects can be
seen as one. They are linked by their pixels as galaxies in
Friends-of-Friends algorithm. A deblending algorithm is then applied to
resolve child objects from their parents (defined as the first detection).
Then a resolve algorithm is applied to extract the best object when multiple
fields are overlapping.

Flags exist for selection with good galaxies. For example, \texttt{FLAGS1}
and \texttt{FLAGS2} flags are combined in a 64 bits mask \texttt{FLAGS} to
know bad object with poor photometry. In the \texttt{Photometry} table,
there is also a \texttt{clean} for a predefined selection of the most common
good flags, and facilitate the selection.

There is many problems with photometry, with cases of bright galaxies with
sky levels not well estimated and missing faint galaxies for example. Most
of these known problems are corrected in the recent releases (DR9 and DR10).

Old releases worked with a spectrograph of 640 fibers, with collisions at
55'', while the new BOSS survey works with a 1000 fibers spectrograph but
with a greater collision size of 64''. The coverage of the old releases
should be used for the new BOSS, so its better to use latest releases.
Moreover, the pipeline used for the spectrum had changed and improved along
releases.

Following definitions given in the SDSS website, we can define two
coordinate systems in the survey.
%
\begin{description}
    \item[Great Circle:] This coordinates system is define with two angles
        $(\mu, \nu)$. Coordinates are relatives to one stripe so they can be
        used when working with galaxies inside a stripe region.

    \item[Survey Coordinates:] It's an other system similar to celestial
        coordinates but ``centred'' on the contiguous block of galaxies  of
        the survey. Coordinates are written $(\lambda, \eta)$. The range of
        these coordinates is: $-\cfrac{\pi}{2}<\eta<\cfrac{\pi}{2}$ and
        $-\pi<\lambda<\pi$.
\end{description}
%
We will work only with survey coordinates as they allow us to easily define
a mask for the SDSS\@. The celestial coordinates and survey coordinates are
the same system of coordinates, except that one is a particular rotation of
the other. The relation between the two systems can be computed and are:

\subsubsection{Survey coordinates to celestial coordinates}

\begin{eqnarray}
    \delta &=&
        \arcsin\left(\cos\lambda\sin\left(\eta+\delta_0\right)\right)
        \nonumber\\
    \alpha &=&
        \mathrm{atan2}
        \left(\sin\lambda,\cos\lambda\cos\left(\eta+\delta_0\right)\right)+
        \lambda_0\nonumber\\
\end{eqnarray}
%
with ${\left(\alpha_0,\delta_0\right)}_{\left(\alpha,\delta\right)}=
{\left(185°,32.5°\right)}_{\left(\alpha,\delta\right)}=
{\left(0,0\right)}_{\left(\lambda, \eta\right)}$.

\subsubsection{Celestial coordinates to survey coordinates}

The inverse transformation is:
%
\begin{eqnarray}
    \eta &=&
        \mathrm{atan2}
        \left(\sin\delta,\cos\delta\cos\left(\alpha-\alpha_0\right)\right)-
            \delta_0\nonumber\\
    \lambda &=&
        \arcsin\left(\cos\delta\sin\left(\alpha-\alpha_0\right)\right)
            \nonumber\\
\end{eqnarray}
%
with ${\left(\alpha_0,\delta_0\right)}_{\left(\alpha,\delta\right)}=
{\left(185°,32.5°\right)}_{\left(\alpha,\delta\right)}=
{\left(0,0\right)}_{\left(\lambda,\eta\right)}$. Periodic conditions must be
applied to angles found by the latter equation:
%
\begin{equation}
    \begin{cases}
        \eta\rightarrow\eta+180° \; \lambda\rightarrow180°-\lambda&
        \mbox{if}\;\eta<-90°\;\mbox{or}\; \eta>90°\\
        \eta\rightarrow\eta-360° &
        \mbox{if}\;\eta>180°\\
        \lambda\rightarrow\lambda-360° &
        \mbox{if}\;\lambda>180°\\
    \end{cases}
\end{equation}
%
\subsubsection{Stripe number}
%
Stripes have a constant width of 2.5° along the $\eta$ coordinate. So,
stripe number $n$ of a galaxy with $\eta$ coordinate is:
%
\begin{equation}
    n = \mathrm{floor}\left(\cfrac{\left(\eta+58.75°\right)}{2.5°}\right)
\end{equation}
%
\subsection{Galaxies selection}
%
Many tables in the SDSS save galaxies and other objects properties extracted
from images of the survey. These tables are the results of different
selections in objects extracted in images. When crossing objects between
images of the survey that overlap, there are some differences of positions
between the same object in the two images. So there are possibilities that
an object is observed twice or more. In many of those tables, there is no
object duplicated.

In the SDSS database, the \texttt{Galaxy} view is a selection from the
\texttt{PhotoPrimary} for objects flagged as \emph{galaxy}. The
\texttt{Galaxy} view contains the photometric parameters (no redshifts or
spectroscopic parameters) measured for resolved primary objects. But we have
other useful informations to link with tables that give us photometric and
spectroscopic redshifts. There is the \texttt{specobjid} to link with
spectroscopic redshifts in the table \texttt{SpecObj} which doesn't contain
duplicates (it's a clean table of \texttt{SpecObjAll} with clean redshifts).
If \texttt{specobjid=0}, the galaxy doesn't have a spectroscopic redshift
(the galaxy wasn't spectroscoped). The \texttt{objid} is a link to the
\texttt{Photoz} table which contains all photometric redshifts for galaxies
in the \texttt{Galaxy} table. Estimation is based on a robust fit on
spectroscopically observed objects with similar colors and inclination
angle. There is also the \texttt{PhotozRF} where estimates are based on the
Random Forest technique. Galaxies in the \texttt{SpecObj} are limited to
$m_r<17.77$ and a given surface brightness. So we need to apply the same
flux limitations when selecting galaxies on the \texttt{Galaxy} table. A
possible \texttt{SQL} query for selecting galaxies in this table and link
them with redshift tables is for spectroscoped galaxies:
%
\begin{listing}[H]
    \begin{minted}[bgcolor=griscode, linenos]{sql}
SELECT GG.ra, GG.dec, GG.petroMag_u, GG.petroMag_g, GG.petroMag_r,
GG.petroMag_i, GG.petroMag_z, GG.specobjid, GG.objid, Z.z, Z.Zerr
FROM Galaxy AS GG, SpecObj AS Z WHERE Z.specobjid=GG.specobjid AND
GG.specobjid!=0 AND GG.petroMag_r<17.77 AND GG.ra<275 AND GG.ra>100 AND
GG.dec>-10 AND GG.dec<75
    \end{minted}
\end{listing}
%
and for galaxies which couldn't be spectroscoped:
%
\begin{listing}[H]
    \begin{minted}[bgcolor=griscode, linenos]{sql}
SELECT GG.ra, GG.dec, GG.petroMag_u, GG.petroMag_g, GG.petroMag_r,
GG.petroMag_i, GG.petroMag_z, GG.specobjid, GG.objid, Z.z, Z.Zerr
FROM Galaxy AS GG, Photoz AS Z WHERE GG.specobjid=0
AND GG.objid=Z.objid AND GG.petroMag_r<17.77 AND GG.ra<275
AND GG.ra>100 AND GG.dec>-10 AND GG.dec<75
    \end{minted}
\end{listing}

Stripe limits are given in the table \texttt{StripeDefs} but this
limits aren't actual, they were planned at the beginning of the survey.

Some planned regions aren't still observed, so we need to define other limits
in $\lambda$ coordinates for not complete stripes. We find by hand the new
limits of stripes which contains spectroscoped galaxies. Now, the survey mask
is like in \bartreffigure{sdss}. We will consider just galaxies in this mask in
order to find groups in the SDSS\@.
%
\begin{figure}[ht] \centering
    \includegraphics[width=\linewidth]{figures/sdss/sdss.png} \caption{Galaxies
    in the SDSS DR10 with stripes limits defined by hand. The red lines limits
of the stripes make the SDSS mask used to identify edges.\label{fig:sdss}}
\end{figure}

\subsubsection{Flags in the SDSS}

Galaxy photometry can have some troubles in the SDSS\@. In the general case,
those objects are flagged with the \texttt{clean} property which indicates by 1
that the photometry is OK and by 0 when there is a problem. Details of the
problems are in the bit flag. But for groups, we need to select all galaxies,
even if they are not clean, or our groups will suffer incompleteness in their
membership and their physical properties such as luminosity, stellar mass\ldots
will be biased. \texttt{Galaxy} table is a selection from \texttt{PhotoPrimary}
view for objects with $\mathrm{\texttt{type}}=3$ (galaxy).

However, we have to take into account the error on the redshift estimation
using the \texttt{zErr}. For photometric redshift, if the \texttt{zErr} is too
high, we can use the \texttt{nnAvgZ} which is the average redshift of galaxies
in the neighbourhood of the considered galaxy. It can be better too if the
photometric redshift is too different from it.

The \texttt{SpecObjAll} contains duplicates and bad datas. But the
\texttt{SpecObj} contains just clean spectras. The field \texttt{zWarning} can
be used to decide if we keep a redshift or not.

\subsection{Fibre collision estimation}

We need a sample of galaxies for which we can easily characterize borders and
where all galaxies, given the flux limit of the survey, are presents. But there
is the problem of fibre collisions galaxies But our algorithm is tested on a
``perfect'' mock catalogue. In order to know the behaviour of the algorithm
with these problematic galaxies, we need to implement this in our mock
catalogue.

In the SDSS, obtaining spectroscopic redshifts of galaxies is done using a
plate of 1.5° diameter, in which there is a certain number of fibres in order
to get spectrum of the galaxy. But in the plate, the number of fibres is
limited. Moreover, each portion of the sky can't be respectroscoped multiple
times, because spectrum are acquired slowly. Although runs may overlap, there
are galaxies that can't be spectroscoped. Indeed, fibres have a dimension of
55''. When galaxies are closer than this distance, one (or more) of those
galaxies aren't spectroscoped. We can see that in the \bartreffigure{plane}
where we have taken the nearest neighbour of a galaxy on the celestial sphere,
and determined the differences in angular size and redshift between the two
galaxies. As expected, the number of galaxies which are closer than 55''
decreases dramatically. There are still some galaxies because the overlapping
of runs can permit to get redshifts for galaxies behind this limit.
%
\begin{figure}[ht] \centering
    \includegraphics[width=0.6\linewidth]{figures/sdss/plane.pdf}
    \caption{\footnotesize{}Distribution of spectroscoped galaxies in the SDSS
    DR8 in angular size and redshift differences with the nearest neighbour
galaxy.\label{fig:plane}} \end{figure}
%
A consequence is that in denser regions, the number of fibre collision
increases, affecting more our groups analysis.

We tried to implement this selection effect in our mock catalogue. For that we
computed the local density in the field, taking all galaxies (spectroscoped or
not) in the neighbourhood of 1.5° of each galaxy, and in the same time, we
determine the fraction of galaxies that don't have a spectroscopic redshift. We
expected to deduce a relation between the density field and the fraction of
fibre collisions. In the mock catalogue, we will compute the same density field
and we apply the relation estimated in the SDSS sample to the mock. We need for
each galaxy to count the fraction of non spectroscoped galaxies in a region of
1.5° radius around. We have to remove galaxies that are close to survey edges,
because if we don't, there are missing galaxies and the fraction will be
affected. The way of selecting those galaxies is to compute a circle of 1.5°
around a galaxy, and if a generated point is out of the survey, the galaxy is
defined as to be closer to the limits.

\remark{%
    We can generate samples of points at an angular distance $d$ to a point of
    coordinate $(\alpha_0,\delta_0)$ using formulas of the spherical triangle.
    If we define a triangle by the pole, the point $(\alpha_0,\delta_0)$ and
    the point whose we want coordinates $(\alpha,\delta)$, we can write the
    following relations using the spherical triangle and its dual:
    %
    \begin{eqnarray}
        \sin\delta&=&\sin\delta_0\cos d + \cos\delta_0\sin d
        \cot\gamma\nonumber\\
        \sin\delta_0\cos\gamma&=&\cos\delta_0\cot{d}-\sin\gamma\cot\left(\alpha-\alpha_0\right)\nonumber\\
    \end{eqnarray}
    %
    where $\gamma$ is like a polar angle, which have all the values between 0
    and $2\pi$. We can rewrite:
    %
    \begin{eqnarray}
        \delta &=&
        \arcsin\left(\sin\delta_0\cos d + \cos\delta_0\sin d\cos\gamma\right)\nonumber\\
        \alpha-\alpha_0 &=& \arctan\left(\cfrac{\sin\gamma}{\cos\delta_0\cot{d}-\sin\delta_0\cos\gamma}\right)\nonumber\\
    \end{eqnarray}
    %
    There are problems at poles. For a $\gamma_0$ limit, angles can't be
    recovered with above formulas. Indeed, the problem appears when
    $\tan\Delta\alpha\rightarrow\infty$. So:
    %
    \begin{equation}
        \cos\delta_0\cot d -\cos\gamma_0\sin\delta_0 = 0
    \end{equation}
    %
    implying:
    %
    \begin{equation}
        \cos\gamma_0=\cfrac{1}{\tan d\tan\delta_0}
    \end{equation}
    %
    So to handle these limit cases, we summarize the correction for the
    differences in right ascensions by:
    %
    \begin{eqnarray}
        \Delta\alpha \rightarrow \Delta\alpha+\pi &\mathrm{if}&
        \mathrm{sign}\left(\delta_0\right)\cos\gamma \geqslant
        \mathrm{sign}\left(\delta_0\right)\cos\gamma_0\nonumber\\
    \end{eqnarray}

    An other way to draw circles in the sphere is to consider the point for
    which we want to know celestial coordinates around a given angular distance
    as the pole of a new coordinate system. In this system, points at given
    distance of our central point are just points with $\pi/2-\delta$ and
    $\alpha$ running between 0 and $2\pi$. We now can determine cartesian
    coordinates of those points in this system and apply a rotation to go from
    the ``real'' system to the system where the central point is the pole.
    This can be easily done if we know the axis of rotation and the angle using
    quaternions, numerically better than Euler angles.
}
%
We didn't see the trend we expected with the density field, so we thought that
it can be due to the large area in which we compute the fraction of
spectroscoped galaxies and we ran the same with a radius of 0.3°, but without
success too.

Moreover, including photometric redshifts in the mock catalogue and in MAGGIE
is very complex. For example, we measured the bias and dispersion of the
distribution of differences between spectrocoped and photometric redshifts in
the SDSS\@. While the dispersion remains roughly constant, the bias increases
with the spectroscoped redshift. So some effects are not still under control
when computing photometric redshifts, and we should avoid their utilization in
galaxy group algorithms when possible. In the case of surveys were
spectroscopic redshifts are not accessible, the photometric redshifts should be
as clean as possible.
%
\begin{figure}[htb] \centering
    \includegraphics[width=0.8\linewidth]{figures/sdss/redshift_difference.pdf}
    \caption{Parameters of a normal distribution for photometric redshifts
    versus spectroscopic redshifts in the
SDSS\@.\label{fig:redshift_difference}}
\end{figure}

\section{Coverage of the SDSS}

For many computations in this thesis, we need to determine the surface covered
on the sky by the galaxy sample used. In the SDSS, the mask we constructed
allows us to do it easily by a Monte Carlo process.

First, we generate a number $N$ of points around a point of coordinates
$(\alpha_0, \delta_0)$ with a maximal angular separation $\theta_{\max}$ which
is larger than the maximal angular separation in our sample. The fraction of
points falling inside the mask gives us the fraction of the generated area
corresponding to the mask. This area is just
$\mathcal{S}=\int_0^{\theta_{\max}}\int_0^{2\pi}\sin\theta\dd{\theta}\dd{\phi}=
2\pi\left(1-\cos\theta_{\max}\right)$. We made this calculation for different
cone angles $\theta_{\max}$ and for different number of points to see if we
have a convergence in the value of the area. Results are shown on
\bartreffigure{sdss_area}.
%
\begin{figure}[htb]
    \centering
    \includegraphics[width=0.8\linewidth]{figures/sdss/SDSS_area}
    \caption{Determination of the area of the SDSS for our selection with a
    Monte Carlo process. Results seem to converge on a value of 2.1993
steradians (roughly $7220°^2$).\label{fig:sdss_area}}
\end{figure}
%
\remark{%
    Generating points uniformly on the celestial sphere around a point of
    coordinates $(\alpha_0, \delta_0)$ to an angular distance $d$ can be
    done by assuming that this point is the upper pole of an other spherical
    system. In this situation, points follow $0\leqslant\theta\leqslant d$
    and $0\leqslant\phi\leqslant2\pi$, assuming spherical coordinates and
    not celestial one. $\phi$ coordinates are generated between the previous
    range. For $\theta$ coordinates, since their distribution isn't uniform
    $\left(f\left(\theta\right)=\cfrac{1}{2}\sin\theta\right)$, they are
    generated by $\theta=\arccos(2U-1)$, where $U$ is a variable following
    an uniform distribution with values between 0 and 1.

    Then, points are rotated by quaternions to $(\alpha_0, \delta_0)$. The
    rotation axis is just the cross product between the pole vector and the
    vector defined by $(\alpha_0, \delta_0)$, and the rotation angle is
    $\cfrac{\pi}{2}-\delta_0$.
}

\section{Galaxy stellar masses}

Galaxy stellar masses available in the SDSS database are not really an
information from the survey. Indeed, contrary to coordinates, magnitudes or
redshifts, the stellar mass is not a direct observable. Its estimation is based
on the application of various models of stellar population on the galaxy
spectrum obtained by the SDSS\@. Several models exist, but they don't provide
the same estimation for a given galaxy. In \bartreffigure{stellar_mass_models},
we compare eight models to have an order of the inaccuracy of the stellar mass:
FSPSGranWideDust, FSPSGranWideNoDust, FSPSGranEarlyDust and FSPSGranEarlyNoDust
from~\cite{Conroy+09}, PassivePort and StarFormingPort from~\cite{Maraston+09},
PCAWiscM11 and PCAWiscBC03 from~\cite{Chen+12} and MPA-JHU~\cite{Brinchmann+04,
Kauffmann+03, Tremonti+04}.

Principal discrepancies between models are coming essentially from the various
stellar population synthesis (SPS) models involved in the fit of the galaxy
spectrum necessary for the stellar mass estimation. But each model has also
some internal variations. For example,~\cite{Conroy+09} assumes an early star
formation in galaxies for its FSPSGranEarlyNoDust (without dust extinction
correction) and FSPSGranEarlyDust (with dust extinction correction), while
FSPSGranWideDust and FSPSGranWideNoDust assume an extended star formation
history. As we can see, differences are relatively important: models using
different SPS have large dispersion in their estimation, while when using the
same SPS, stellar masses are coherent. Some models are also biased between each
other, but bias can be corrected and not considered in our analysis. Generally,
models agree to better than 0.3 dex, i.e.\ errors on individual masses are of
$0.3 / \sqrt{2} = 0.2$ dex. In particular, the MPA-JHU masses agree with all
others to typically better than 0.2 dex in $\sigma$.

\begin{figure}[htb]
    \centering
    \includegraphics[width=\linewidth]{figures/sdss/stellar_mass_models.pdf}
    \caption{Comparison between stellar mass models applied onto galaxies from
        SDSS\@. Contours are showing number levels in unit of $\sigma$.
        Ordinates and abscissas are the stellar masses of galaxies in decimal
        logarithm of the solar mass. The upper left box shows the bias and
        dispersion of the logarithmic difference between both models. Models
        are: FSPSGranWideDust, FSPSGranWideNoDust, FSPSGranEarlyDust and
        FSPSGranEarlyNoDust from~\cite{Conroy+09}, PassivePort and
        StarFormingPort from~\cite{Maraston+09}, PCAWiscM11 and PCAWiscBC03
    from~\cite{Chen+12} and MPA-JHU~\cite{Brinchmann+04, Kauffmann+03,
Tremonti+04}\label{fig:stellar_mass_models}}
\end{figure}

\section{Final galaxy sample}
\label{sec:final_galaxy_sample}

All previous sections are showing something important in the SDSS data:
observational errors are important, and the automatic processing of this data
sometimes leads to false detections, artefacts\ldots, making analysis and
corrections complex.

Fortunately, recently~\cite{Tempel+14} in its FoF analysis of galaxy groups in
the SDSS-DR10 had to deal too with such problems and the contamination they
introduce. Major problems are stars classified as galaxies, nearby large
galaxies resolved on several galaxies or poor photometry of some galaxies due
to bright stars or bad sky level estimation in the neighbourhood. They
performed an incredible filtering on the sample by visually checking 30000
galaxies that were potential problematic galaxies. They checked (extracted from
\cite{Tempel+14}):
%
\begin{itemize}
    \item $10000$ apparently brightest galaxies (in r-band). For galaxies
        brighter than m r < 13.5 about 10\% of the objects were spurious. For
        galaxies 13.5 < M r < 14.5, about 1\% were spurious entries; this
        fraction decreases with luminosity;
    \item $5000$ intrinsically brightest galaxies in the sample (< 1\% were
        spurious);
    \item $3000$ intrinsically faintest galaxies in the sample (to ensure the
        correctness of the faint-end of the luminosity function);
    \item all the sources with the spectroscopic class QSO\@;
    \item all the objects with bestobjid missing or not GALAXY\@. For these
        objects, we used fluxobjid if the matched photometric object was
        classified as a galaxy;
    \item all the objects for which the difference between r-band point spread
        function (PSF) magnitude and model magnitude was smaller than 0.25
        (thus further excluding some of the stellar sources in the catalogue);
    \item all the galaxies with the difference between r-band Petrosian and
        model magnitudes greater than 0.4;
    \item all the galaxy pairs that were closer than 5' (in order to remove
        double/multiple entries);
    \item the entries where the colour indices g−r, r−i, and g−i had extreme
        values.
\end{itemize}

Finally, the sample was cleaned for around $600$ galaxies and $1400$ of
galaxies were flagged as having a bad photometry. We decided to use directly
the galaxy sample they obtained since it covers exactly the same area we use
and their conscientious clean up of the SDSS-DR10 is amazing and very helpful.

% vim: set tw=79 :
