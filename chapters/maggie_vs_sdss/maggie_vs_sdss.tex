\bartchapterimage{heic0911b.jpg}
\chapter[MAGGIE vs SDSS]{MAGGIE versus SDSS}
\label{cha:MAGGIE_vs_SDSS}
\bartthumb{heic0911b.png}

\minitoc%

\section{Introduction}
\label{sec:vs_introduction}

MAGGIE is designed for an optimal extraction of galaxy groups in several
galaxy surveys such as the SDSS\@. Once the group catalogue in our
possession, we are able to analyze galaxies in both local and global
environments. Our study of environment is motivated by~\cite{Peng+10}
results, showing that with their estimation of the environment of galaxies,
the star formation rate (SFR) is independent of it, except for high mass
galaxies where the SFR is lower in dense environment. But as discussed in
the introduction of the thesis, the tracer used for the environment doesn't
distinguish between both environments. With our results, we make the same
analysis to see modulations with the projected radius in virial units and
the host halo mass.

\section{Environment modulation}
\label{sec:environment_modulation}

\cite{Peng+10} has shown that the mean SSFR is not very dependent of the
environment, using the over-density as a tracer. But we can go further and
see its modulation with local and global environment through the projected
radius of the galaxy in its halo and the virial mass of the host halo.

We tried this with both MAGGIE and the group catalogues provided
by~\cite{Tempel+14} for a simple comparison.

\subsection{MAGGIE}
\label{sub:ssfr_maggie}

