\bartchapterimage{simudm}
\chapter{Generate mock catalogues}
\label{cha:mock}
\bartthumb{thumb_simudm}

\minitoc%

\section{Introduction}

A mock catalogue is an useful tool to test algorithms involving galaxies in
order to see if it is operational in a realistic situation. Many of the
properties of galaxy surveys can be simulated: the spatial clustering of
galaxies, luminosity function, incompleteness and measures errors are some
examples of them. There are different methods to obtain such a mock catalogue.
All of them involves cosmological simulations and there halos of dark matter.
According to the model of galaxy formation, we can use halo occupation
distribution (HOD) to populate dark matter haloes with galaxies and putting
some luminosity functions (for example) as constraints. We can follow galaxies
in semi-analytical models (SAM) in those cosmological simulations outputs in
order to have statistical properties of galaxies which agree with observational
results. With such realistic galaxies, we can use those simulation boxes to
place an observer into it, and create a mock survey. But to have a realistic
mock catalogue, it's necessary to take care of many things which will be
described in the next section.

\section{Mock structure}

In all this section, we will assume that we have already in our possession a
dark matter simulation box which has been populated with galaxies with one of
the methods described below (SAM, HOD\ldots). At this step, physical properties
of those galaxies aren't interesting.

\subsection{Placing boxes}

The first step to make a mock catalogue is to get galaxies positions like in a
survey, to get an $(\alpha,\delta)$ frame to simulate the sky coverage of
survey.

The mock catalogue must have the same volume as the galaxy survey we want to
mimic. For example for the SDSS survey, we can measure redshift to a value of
0.3 and more. But the problem is that the majority of the simulation boxes have
a size of around $L_{\mathrm{box}}=100-300 h^{-1}$ Mpc, letting us with a
maximal redshift in our false survey of around
${H_0}{L_{\mathrm{box}}}/c\approx 0.025$ in the case of a box of $100 h^{-1}$
Mpc sized. Bigger simulations exist, and allow us to access higher redshifts,
but this increasing size reduces the resolution of the simulation in particle
mass and therefore we can't have low mass halos in the simulations.

The solution is to take a ``little'' simulation box and to replicate it and to
make some ``Tetris'' cube until we reach the maximal redshift we want. An
example of the resulting ``mock cube'' is shown on \bartreffigure{cubemock}.
%
\begin{figure}[htb]
    \centering
    \includegraphics[width=0.5\linewidth]{figures/mock/mock}
    \caption{The structure of the mock catalog once we have replicated the
        simulation box chosen to populate dark matter halos. Each cube
        represents a simulation box whose galaxies were randomly rotated and
        translated in positions. Placing an observer at a given position (the
    black dot), we can access different geometry for the survey and go to
higher redshift ranges than those possible with an unique simulation
box.\label{fig:cubemock}}%
\end{figure}

Now if we take an observer at some position into this big box, we can have
different sky coverage for the observer. The simplest is to place the observer
at a corner, which gives a solid angle of $\pi/2$ steradians. At the centre,
we have a full sky coverage but we reduce the redshift extension by two. For
the SDSS, as in \bartreffigure{cubemock}, the area of the survey is large (see
\bartrefappendix{sdss}) and we need to cover half of the sky to get the same
volume.

If we want to care about redshift evolution of galaxies for the observer, we
need to use other snapshots at different redshifts, simply joining cubes in
comoving coordinates. Indeed, the cosmological redshift of the galaxy, only
consequence of the Hubble flow, is deduced from the relation between the
redshift and the comoving distance, equals to the comoving transverse distance
(or proper motion distance) in the case of a flat Universe ($\Omega_k=0$).
Moreover, the comoving separation $R_c$ between two points with angular
separation $\theta$ on the sky, at comoving distance $D_c$ from the observer,
are simply related by a geometrical relation $R=\theta D_c$. This separation
$\theta$ deduced from comoving coordinates should be the same as those of the
observer working with physical coordinates. The observer wants to know the
physical separation $R_p$ between the two galaxies, so $R_p=\theta
d_\mathrm{ang}$ giving $R_p \left(1+z\right)= R_p/a\left(t\right)=R_c=\theta
D_c$.

Placing boxes as described previously creates a perspective effect from the
point of view of an observer \citep{Blaizot+05}, and the consequences aren't
predictable in a statistical sense. To avoid this, we apply some
transformations on galaxies in the initial cube like inversions, rotations and
periodic translations. Rotations are multiples of $\pi/2$ around the three
principal coordinates axes, because if other rotations are allowed, this create
over-densities in some regions of the final mock which aren't physical.
Translations are performed on the three principal axes and when galaxies are
out of the initial cube, periodic conditions are applied. All of those
transformations are randomly generated for each cube in the final mock
catalogue.

\subsection{Physics}

\subsubsection{Celestial coordinates}

The first step to simulate this is to transform Cartesian coordinates in the
3D space to celestial coordinates ($(\alpha,\delta)$ frame). Getting these
coordinates is the same as computing spherical coordinates.
%
\begin{equation}
    \alpha=\left\{ \begin{array}{lcr}
     \mbox{arctan2} \left(Y, X\right)+2\pi & \mbox{if} & Y>0 \\
     \mbox{arctan2} \left(Y, X\right) & \mbox{else} & \\
    \end{array}\right.\nonumber%
\end{equation}
%
\begin{equation}
    \delta=\mbox{sign} \left(Z\right)
    \arccos\left(\frac{\sqrt{X^2+Y^2}}{\sqrt{X^2+Y^2+Z^2}}\right)
\end{equation}

\subsubsection{Redshifts}

In our case, the origin of coordinates is the observer. If we keep the distance
as calculated previously, the observer can still have precise determination of
the distance of a galaxy. In reality, we observe it in redshift space so the
redshift as distance indicator is biased by peculiar velocities. Our initial
galaxy catalog allows us to get the velocity of a galaxy, so we ompute the line
of sight (los) velocity of this galaxy relatively to the observer.
%
\begin{equation}
v_\mathrm{los}=\cfrac{\textbf{OG}.\textbf{v}_\mathrm{pec}}
    {\left|\left|\textbf{OG}\right|\right|}
\end{equation}
%
where $O$ is the observer and $G$ the galaxy, $\textbf{v}_{\mathrm{pec}}$ its
peculiar velocity. This velocity has a sign. The redshift is just the
expression of a shift in wavelength. The observed wavelength $\lambda$ is
linked to the original (emitted) wavelength $\lambda_0$ by:
%
\begin{equation}
    \lambda=(1+z)\lambda_0
\end{equation}
%
The shift caused by Universe expansion is
$\lambda_{\cos}=(1+z_{\cos})\lambda_0$ where the subscript $\cos$ refer to the
cosmological expansion. The shift caused by the peculiar velocity is
$\lambda=(1+z_{\mathrm{pec}})\lambda_{\cos}$. So the observed wavelength is
$\lambda=(1+z_{\mathrm{pec}})(1+z_{\cos})\lambda_0$. The resulting observed
redshift is just:
%
\begin{equation}
    (1+z)=(1+z_{\mathrm{pec}})(1+z_{\cos})
\end{equation}
%
The peculiar redshift is the just the relativist Doppler effect:
%
\begin{equation}
    (1+z_{\mathrm{pec}})=\sqrt{\cfrac{1+\beta}{1-\beta}}
\end{equation}
%
with $\beta={v_{\mathrm{los}}}/{c}$. The cosmological redshift is
approximated by $z_{\cos}={H_0}{D}/c$ where $D$ is the physical distance of
the galaxy to the observer and $H_0$ the Hubble constant.
%
\comments{I think we need to add the velocity of the Local Group in the
redshift, because in our case the observer has a null velocity. Maybe
corrected in SDSS data?}

Applying this method to mock catalogue, we can have galaxies whose distance
is biased by peculiar velocities in redshift space. With such a treatment,
the velocity dispersion of galaxies in groups leads to the apparition of
``fingers of God'' as seen in observations in redshift space.

\subsubsection{Survey mask}

With our frame in redshift space relative to the observer, we can apply
different masks on angular coordinates according to the survey we want to
mimic. An example of such a mask is in \bartrefappendix{sdss}, where we
describe how to decide if a galaxy is inside the mask or not.

\subsubsection{K-corrections}

In reality, an observer study galaxies in a given bandwidth in wavelength and
can't use the bolometric flux of the object. With the expanding Universe, all
the spectral energy distribution (SED) of galaxy is shifted. All wavelengths
are shifted by the same value for a given redshift. So, knowing the luminosity
$L$ of a galaxy in a given band in reality (using the true SED), computing its
apparent magnitude for an observer aren't as easy as correcting for the
distance modulus. The observer in a given band sees a different part of the
rest frame SED\@. The flux observed in the same band as the rest frame flux is
maybe higher or lower. A correction for this effect is needed in real galaxy
survey to estimate the distance of an object and must be taken into account in
our mock catalogue.

As explained before, this correction depends on the SED of galaxies and the
band used in the survey. The common way of correcting, it when we have a
multi-band photometry, is to fit the observed SED in those bands with
theoretical templates of SEDs. Such templates can be obtained with existing
programs as PEGASE \citep{LeBorgne+04}, which give us galaxy SEDs. But those
programs are a little time consuming, a problem for mock when we want to run
several of them. A good solution is provided by \citet{Chilingarian+10}, where
the K-correction is fitted on templates for SED as given by PEGASE in terms of
a polynomial of the redshift of the galaxy and its colour. The corresponding
K-correction is precise for redshifts until 0.3 in different survey bands
(including $ugriz$ for the SDSS). This work reduces the computation of
K-corrections to the use of simple polynomial relations and make our task
easier.

By definition, the K-correction $K$ for a galaxy of apparent magnitude $m_X$
in a given band $X$ and absolute magnitude $M_X$ in the same band is:
%
\begin{equation}
    m_X=M_X + 5\log_{10}\left(d_\mathrm{lum}\left[pc\right]\right) - 5 + K
\end{equation}
%
In our case, the K-correction depends on the redshift of the galaxy and its
colour in apparent magnitude given two bands. So we can rewrite:
%
\begin{equation}
    \label{eq:appmag}
    m_X=M_X +
        5\log_{10}\left(d_\mathrm{lum}\left[pc\right]\right) - 5 +
        K\left(z, m_X - m_{X'}\right)
\end{equation}
%
where:
%
\begin{equation}
    K\left(z,m_X-m_{X'}\right)=\sum_{i=0}^{N_i}\sum_{j=0}^{N_j} a_{ij} z^i
    {\left(m_X-m_{X'}\right)}^j
\end{equation}
%
and $a_{ij}$ is a $N_i\times N_j$ matrix containing the coefficients of the
two dimensional polynomial. These coefficients depend on the bands of the
survey used for the colour computation.

The observer in the mock can just, in theory, access to apparent magnitude of
the survey. But we don't know in advance these magnitudes, and as we can see in
the expression of \bartrefequation{appmag}, we need apparent magnitudes to
compute apparent magnitudes. If we use the other bands of the survey, with
$a_{ij}$ coefficients, we can always write a set of equations for a galaxy
which involves all apparent magnitudes of the survey. So we can write a set of
non linear equations with polynomial of order $N_j$ (redshift of the galaxy is
supposed to be known). Numerically it's easy to solve this set of equations,
and relatively fast with equations solvers or by iterations. In practice, the
first is faster than the second method, even if both methods give similar
results in apparent magnitudes.

\subsubsection{Flux limit}

We have seen in \bartrefappendix{sdss} that spectroscoped galaxies are just
defined for galaxies whose apparent magnitude is less than 17.77 in $r$ band.
So, in all the redshift sample, we miss some galaxies not sufficiently bright.
To take into account this effect, we remove galaxies not reaching the limit
apparent magnitude of the survey. An additional selection on surface
brightness is also done in the SDSS, but estimating them is difficult from
virtual galaxy catalog and the number of ``lost'' galaxies is sufficiently low
to ignore this step in the construction of the mock catalog.

\subsubsection{Spectroscopic and photometric redshifts}

Sometimes, we can't access to spectroscopic redshifts, more precise than
photometric redshifts. In the SDSS, for example, this is due to tiling process.
Fibers analysing the spectrum of galaxies can't be closer from each other than
55'', so if for a target galaxy (selected to obtain a spectrum) there is an
other galaxy closer than those 55'', the tile containing all fibers doesn't
have the possibility to measure the redshift of this galaxy. This problem is
more significant for dense regions in the celestial plane. A very good
algorithm to place tiles in order to limit the number of missed galaxies (i.e.\
the number of fiber collision) has been applied in the galaxy sample of the
SDSS \citep{Blanton+03}.

But there is still some galaxies without spectroscoped redshifts. If we remove
those galaxies from our sample, there will be a spectroscopic incompleteness
with unknown effects on our results.

Unfortunately, there is no simple way to simulate this in the mock catalogue
and we choose to ignore it. Just a small fraction of galaxies are not
spectroscoped and this must not affect our results.

\subsubsection{Observational errors}

The way we organized the construction of the mock catalogue is useful for the
introduction of observational errors. For example, we treat the case were we
want to add errors on the absolute magnitude of galaxies in the final mock
catalog. If we have a model for introducing such errors according to some
physical galaxy properties in the virtual galaxy catalogue, we can just add them
inside this virtual catalogue and magnitude errors will be reported on the mock galaxy
catalogue. If errors depend on properties computed in the mock catalogue, we
can simply add magnitude errors while constructing the mock catalogue. Any
kind of errors can be added such as redshift measurement errors, astrometry,
photometry, \dots

\subsection{Galaxy samples}
\label{sub:galaxy_samples}

\subsubsection{Definition}
\label{ssub:galaxy_sample_definition}

All previous steps lead to a final galaxy mock catalogue, with or without
observational errors, flux limited for a given apparent magnitude. But working
with flux limited samples give the obligation to correct for missing galaxies
when extracting galaxy group. The only way to avoid errors introduced by the
different choices of the model is to work with a doubly complete sample of
galaxies: limited in luminosity and in volume to not miss any galaxy in our
sub-sample.
%
\begin{figure}[htb]
    \centering
    \includegraphics[width=0.8\linewidth]{figures/mock/subsamples.pdf}
    \caption{An illustration of the doubly complete galaxy samples used. Black
        dots represents galaxies in the mock catalogue. Colored rectangles show
        samples for a threshold magnitude in the $r$ band $M_r$ of -19, -20,
        -21. Their sizes reflects the corresponding limits of the minimal
    luminosity. As we can see, in these regions, no need to correct for missing
galaxies. All galaxies above the given threshold magnitude are visible.
\label{fig:subsamples}}
\end{figure}

We choose a minimal galaxy luminosity for our sample and the maximal redshift
is computed with the maximal distance at which we can observe a galaxy with
this minimal luminosity. We note that if K-correction is considered, this
limit isn't not certain and depends on the considered galaxy. A clean
definition of the sample in this situation should be done by restricting a
little more the redshift extent to not loose fainter galaxies. But the galaxy
loss is low and we didn't consider such a case. In \bartreftable{samples}, we
show the six galaxy sample we constructed from our flux limited galaxy mock
catalogue, with statistics on each of them.
%
\begin{table}[htb]
    \caption{Doubly complete mock galaxy subsamples\label{tab:samples}}
    \begin{center}
    \setlength{\tabcolsep}{3pt}
    \begin{tabular}{lccccccc}
    \toprule
    \toprule
    ID & $M_r^{\max}$ & $L_r^{\min}/L*$ & $z_{ \max }$ & Number & $n$
    & $n^{-1/3}$ & Fraction\\
              &            &     &                 &        & ($\rm  Mpc^{-3}$)
    & (Mpc) & split\\
    \toprule
    1 & --18.5 & 0.09 & 0.042 & \ \,47158 & 0.0125 & 4.32 & 5.3\%\\
    2 & --19.0 & 0.14 & 0.053 & \ \,72510 & 0.0099 & 4.66 & 6.1\%\\
    3 & --19.5 & 0.22 & 0.066 & 112629    & 0.0078 & 5.05 & 6.6\%\\
    4 & --20.0 & 0.36 & 0.082 & 166899    & 0.0058 & 5.56 & 7.4\%\\
    5 & --20.5 & 0.56 & 0.102 & 213546    & 0.0040 & 6.29 & 8.6\%\\
    6 & --21.0 & 0.90 & 0.126 & 245821    & 0.0025 & 7.40 & 9.9\%\\
    \bottomrule
    \end{tabular}
    \end{center}
\parbox{\hsize}{Notes: Columns are: sample, maximum $r$-band absolute
    magnitude, minimum luminosity in units of $L*$ (adopting $M*=-20.44 +
    5\,\log h$ in the SDSS $r$ band from \citealp{Blanton+03}), maximum
    redshift, sample size, mean density $n$, proxy for the mean separation to
    the closest neighbor, $n^{-1/3}$, and the percentage of true groups that
    are flagged because they are split during the simulation box
    transformations. The minimum redshift of each subsample is $z=0.01$.
}
\end{table}

\subsubsection{Limitations}
\label{ssub:galaxy_samples_limitations}

The mock catalogue is constructed from the adjoining of multiple simulation
boxes, each of them having periodic boundaries. A consequence is that some
galaxy groups are split by a simulation box size, and from the point of view of
the observer, members are at two different locations on the celestial sphere.
Inclusion of such groups in statistics leads to biased results of the
performance of grouping algorithms, and a flag is used to distinguish and
remove such groups.

Moreover, the limited volume extension of the survey truncates some groups
close to limits. In the redshift space, these limits are of two kinds: the
angular mask cutting groups all along the line-of-sight, and the redshift cut
with a more important effect due to the elongation in redshift coordinates by
the intrinsic velocity dispersion of the system. Since all the information on
the group is not accessible by the observer, estimation of group properties
less precise. To avoid the degradation of galaxy group performances, we flag
selected groups (after the application of a grouping algorithm) if they are
close to edges (see \bartrefchapter{friends_of_friends_algorithm}) and remove
them from the statistics in tests.

% vim: set tw=79 :
