\bartchapterimage{potw1252a.jpg}
\chapter{Conclusions and perspectives}
\label{cha:conclusions_and_perspectives}
\bartthumb{potw1252a.png}

\section{Conclusions}
\label{sec:conclusions}

An optimal extraction of galaxy groups from the redshift space is not an easy
task. The observer has to deal with a lot of observational errors, projection
effects and bias to perform such an optimal grouping. We shown that all
previously created galaxy group algorithms are imperfect in the sense that with
their assumptions, there are some lacks in extracted groups, explaining the
apparition of two different kind of algorithms: Bayesian and geometrical.

We constructed a galaxy mock catalogue to test several galaxy group algorithms,
whose Friends-of-Friends algorithm to understand what is the optimal set of
linking length to use for galaxy group extraction. We conclude that the values
to use depends on the science to do with these groups.

We developed MAGGIE, a Bayesian galaxy group finder, using probabilities to
constrain the membership in groups. We show by tests on mock catalogues that
MAGGIE performs better than the purely geometrical approach of the percolation
algorithm, even in the case where the galaxy data is polluted by observational
errors. By this way, we also show that Bayesian algorithms are more sensible to
the quality of observational data than geometrical ones as FoF, but that they
allow to recover groups properties and their membership better than FoF.

The application of MAGGIE on real galaxy surveys implies a full understanding
of the possible incompletenesses whose surveys are suffering. The analysis of
the SDSS-DR10 indicates that correcting for luminous and spectroscopic
incompletenesses is very important but also very difficult, since the
extraction of galaxy groups implies no missing galaxies. Lacks in the
membership can affect the grouping, but also the informations obtained from
their analysis. Indeed, environmental effects we want to observe from these
galaxy groups (essentially through the SSFR on which we can't apply any
correction) can be very different if not taking them into account. MAGGIE is
very powerful tool for galaxy group analysis, but we have to apply it carefully
on the analysed data or the interpretation of results can be biased.

\section{Perspectives}
\label{sec:perspectives}

We plan to run MAGGIE on the SDSS-DR10 and publish optimized galaxy groups in
different doubly complete subsamples in redshift and luminosity, and to
re-assess the modulation of sSFR, etc\ldots with local and global environments.
If a modulation of galaxy properties is observed, we will model it and apply it
in semi-analytical codes to see if it reduces discrepancies between
observations and outputs of such codes. It will be a new measure of quenching
with global and local environments. We also wish to run MAGGIE on the deeper
GAMA redshift survey to be able to extract the evolution in time of
environmental effects on galaxies.

Still for MAGGIE, we plan to improve the galaxy grouping by the use of the
red-blue segregation of galaxies. We can use some priors for the modulation of
the fraction of blue galaxies in groups to adapt the probability computation
according to the class of the galaxy. Then, we can iteratively reduce the
impact of our initial model for the blue fraction by using the informations
obtained by MAGGIE to de-project the red-blue segregation observed in groups.
For next iterations, we can re-use our new real space model in the probability
computation, and do it until the convergence of memberships.

In parallel, we plan to launch a collaborative project with other galaxy group
algorithms developers. We will propose to each developer (and myself) to apply
algorithms on a set of mock catalogs constructed in the same way to avoid
cosmic variance on the results, for blind tests. Then, we run the same tests on
each algorithm in order to have a clear understanding of the strengths and
weaknesses of each of them. It will be the first time that galaxy grouping
algorithms will be compared in the same conditions.

It is also interesting to know if there is a limit to recover the real space
modulation of galaxy properties (such as specific star formation rate) with
environment when trying to extract it from projected redshift space. This can
be easily done by imposing ourselves a modulation in the outputs of galaxy
formation codes, and then construct galaxy mock catalogues in redshift space.
Then, using real space information, we can see if the imposed modulation is
recovered, and if galaxy group algorithms introduce biases in some cases.

In continuation with the thesis work, try to theoretically explain the observed
dependency of galaxy properties with their environments. Using hydrodynamical
simulations of galaxies in groups, we wish to understand and model
intra-cluster physical processes (ram pressure stripping, tidal
stripping\ldots). This will imply running academic simulations independently
for each physical process, then model as a function of the different input
parameters (local and global environment essentially). And by trying to switch
them on-off in semi-analytical codes of galaxy formation, determine their
relative importance on galaxy properties (sSFR, bulge to disk ratios\ldots).

% vim: set tw=79 :
