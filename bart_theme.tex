% To define all commands relative to the bart theme for thesis
% commands for fancy
\definecolor{caption}{RGB}{245,222,179}
\captionsetup[figure]{box=colorbox,boxcolor=caption,slc=off}
\let\oldcaption=\caption%
\renewcommand{\caption}[1]{{\oldcaption{\footnotesize#1}}}
%
% useful commands
\newcommand{\ifempty}[3]{\ifx#1\empty#2\else#3\fi}
%
%%%%%%%%%%%%%%%%%%%%%%%%%%%%%%%%%%%%%%%%%%%%ù
% Define the geometry of the thesis
%%%%%%%%%%%%%%%%%%%%%%%%%%%%%%%%%%%%%%%%%%%%ù
\geometry{inner=1.5cm, outer=3.5cm, includefoot=true}
\geometry{headsep=0cm, headheight=53pt}
\geometry{footskip=-3cm}
%
% Put an image at the start of chapter
%%%%%%%%%%%%%%%%%%%%%%%%%%%%%%%%%%%%%%
% First start by defining a command with title of chapter
% in argument to create the chapter
\newcommand{\thechapterimage}{}
\newcommand{\bartchapterimage}[1]{\renewcommand*{\thechapterimage}{#1}}
\newcommand{\bartchapterheadfont}{\fontfamily{ugq}\fontsize{24pt}{30pt}\selectfont}
\newcommand{\newchaptercmd}[1]{%
    \begin{tikzpicture}[overlay,remember picture]
        \begin{scope}
            \clip(current page.north west) rectangle ($ (current page.north east) + (0cm,-9cm)$);
            \fill[black!20] (current page.north west) rectangle
            ($ (current page.north east) + (0cm,-9cm)$);
            \node[below=-0.35cm] (block) at (current page.north)
            {\ifempty{\thechapterimage}{}{\includegraphics[height=9cm, width=\paperwidth]{\thechapterimage}}};
            %\node[fill=red!60!blak] at (block.south) {\printcontents[chapters]{}{1}{}};
        \end{scope}
        \begin{scope}
            \node[right=2cm,rectangle,draw,color=white,fill=white,opacity=0.7,inner ysep=10pt,
            inner xsep=15pt,rounded corners=0.75cm] at ($ (current page.north west) + (0cm,-7cm)$)
            {\bartchapterheadfont\vphantom{pb}\phantom{#1}\hspace{\paperwidth}\null};
            \node[right=2cm,inner ysep=10pt,inner xsep=15pt] at ($ (current page.north west) + (0cm,-7cm)$)
            {\bartchapterheadfont\vphantom{pb}\color{red!60!black}#1};
            \node[left=15cm,rectangle,draw,color=white,fill=white,opacity=0.7,inner ysep=10pt,
            inner xsep=15pt,rounded corners=0.25cm] at ($ (current page.north east) + (0cm,-1cm)$)
            {\bartchapterheadfont\vphantom{pb}\phantom{\chapterlabel}\hspace{5.5cm}\null};
            \node[left=15cm,inner ysep=10pt,inner xsep=15pt] at ($ (current page.north east) + (0cm,-1cm)$)
            {\bartchapterheadfont\vphantom{pb}\color{red!60!black}{\centering\chaptertitlename~\chapterlabel}};
        \end{scope}
    \end{tikzpicture}
}
% now specify format
\titleformat{\chapter}
    {\gdef\chapterlabel{}}
    {\gdef\chapterlabel{\thechapter.\ }}
    {0pt}
    {\newchaptercmd}
    {}
\titlespacing*{\chapter}{0pt}{0pt}{5cm}
%%%%%%%%%%%%%%%%%%%%%%%%%%%%%%%%%%%%%%%%%%%%%%%%%%%%%%%%%%%%%%%%
% To place the page number according to the parity of the page
%%%%%%%%%%%%%%%%%%%%%%%%%%%%%%%%%%%%%%%%%%%%%%%%%%%%%%%%%%%%%%%%
\newlength{\lsquare}% square length
\setlength{\lsquare}{\marginparwidth}% impose square length
\makeatletter%
\newcommand{\thethumbnail}{}
\newcommand{\bartthumb}[1]{\renewcommand*{\thethumbnail}{#1}}
\newcommand{\pagenumber}{%
    % The text for the number to appear
    \newcommand{\inter}{{\fontsize{37}{37}\bf\color{red!60!black}{\thepage}}}%
    \begin{tikzpicture}[overlay,remember picture]
        \begin{scope}
        \ifodd\c@page{%
            \foreach \myi/\op in {0/1.,1/0.8,2/0.6,3/0.4,4/0.2} {%
                \node[opacity=\op] at ($ (current page.north east) + ( -0.5\lsquare-\myi\lsquare, -0.5\lsquare ) $)
                {\ifempty{\thethumbnail}{}{\includegraphics[height=\lsquare, width=\lsquare]{\thethumbnail}}};
                \node[opacity=\op] at ($ (current page.north east) + ( -0.5\lsquare-\myi\lsquare, -0.5\lsquare ) $)
                {\bartchapterheadfont\vphantom{pb}\color{red!60!black}\thepage};
            }%
        }%
        \else{
            \foreach \myi/\op in {0/1.,1/0.8,2/0.6,3/0.4,4/0.2} {%
                \node[opacity=\op] at ($ (current page.north west) + ( 0.5\lsquare+\myi\lsquare, -0.5\lsquare ) $)
                {\ifempty{\thethumbnail}{}{\includegraphics[height=\lsquare, width=\lsquare]{\thethumbnail}}};
                \node[opacity=\op] at ($ (current page.north west) + ( 0.5\lsquare+\myi\lsquare, -0.5\lsquare ) $)
                {\bartchapterheadfont\vphantom{pb}\color{red!60!black}\thepage};
            }%
        }\fi
        \end{scope}
    \end{tikzpicture}
}%
\makeatother%
%
%%%%%%%%%%%%%%%%%%%%%%%%%%%%%%%%%%%%%%%%%%%%%%%%%%%%%%%%
% To set the filigran of the thesis
%%%%%%%%%%%%%%%%%%%%%%%%%%%%%%%%%%%%%%%%%%%%%%%%%%%%%%%%
\makeatletter%
\newlength{\hh}
\newlength{\ww}
\newcommand{\filigran}[2]{%
    % The text for the number to appear
    \settoheight{\hh}{{\fontsize{130}{150}\bf\color{#2}#1}}
    \settowidth{\ww}{{\fontsize{130}{150}\bf\color{#2}#1}}
    \begin{tikzpicture}[overlay,remember picture]
        \begin{scope}
        \ifodd\c@page{%
            \node[shift={(-0.49\hh,0.51\ww)},rotate=90] at
            ($ (current page.south east) $)
            {{\fontsize{130}{150}\bf\color{black!25}#1}};
            \node[opacity=1,shift={(-0.5\hh,0.5\ww)},rotate=90] at
            ($ (current page.south east) $)
            {{\fontsize{130}{150}\bf\color{#2!35!white}#1}};
        }%
        \else{
            \node[shift={(0.49\hh,0.49\ww)},rotate=-90] at
            ($ (current page.south west) $)
            {{\fontsize{130}{150}\bf\color{black!25}#1}};
            \node[opacity=1,shift={(0.5\hh,0.5\ww)},rotate=-90] at
            ($ (current page.south west) $)
            {{\fontsize{130}{150}\bf\color{#2!35!white}#1}};
        }\fi
        \end{scope}
    \end{tikzpicture}
}%
% Now apply it on all pages
\AddToShipoutPicture{%
    \filigran{THESIS}{vior}%
}%
\makeatother
%%%%%%%%%%%%%%%%%%%%%%%%%%%%%%%%%%%%%%%%%%%%%%%%%%%%%%%%%%%%
% To define style of header en footer
%%%%%%%%%%%%%%%%%%%%%%%%%%%%%%%%%%%%%%%%%%%%%%%%%%%%%%%%%%%%
\makeatletter
%
% Define the style of odd pages with chapters
\newcommand{\header}[5]{%
    \ifthenelse{\equal{#1}{}}{}{%
        \begin{tikzpicture}[remember picture]
            \node[fill=#2,opacity=#4,drop shadow,rounded corners=#5,shift={(0.,1em)},inner sep=10pt] at (0.,0.)
            {{\color{#3}#1}};
    \end{tikzpicture}
    }%
}
\fancypagestyle{plain}{
    \fancyhead{}
    \fancyfoot{}
    \renewcommand{\headrulewidth}{0pt}}
%
\fancypagestyle{these}{%
    \fancyhf{}
    \fancyhead[LE]{\pagenumber\header{\rightmark}{bleuf}{white}{1}{0.3cm}}%
    \fancyhead[RO]{\pagenumber\header{\leftmark}{vior}{white}{1}{0.3cm}}%
    \fancyfoot[LO]{}
    \fancyfoot[RE]{}
}%
\makeatother
%
%%%%%%%%%%%%%%%%%%%%%%%%%%%%%%%%%%%
% Set the page style
%%%%%%%%%%%%%%%%%%%%%%%%%%%%%%%%%%%
\pagestyle{these}
\renewcommand{\chaptermark}[1]{\markboth{\textbf{\chaptertitlename~\thechapter.}~#1}{}}
%\renewcommand{\sectionmark}[1]{\markright{#1}}
%
%%%%%%%%%%%%%%%%%%%%%%%%%%%%%%%%%%%%%%%%
% To put a note with a note style
%%%%%%%%%%%%%%%%%%%%%%%%%%%%%%%%%%%%%%%%
\definecolor{couleurPostIt}{rgb}{.9,.9,.35}
\tikzset{fondPostit/.style={color= couleurPostIt}}
\tikzset{ombrePunaise/.style={color={blue!10!gray}}}
\tikzset{ombrePostit/.style={color={black},opacity=.5}}
\tikzset{punaise/.style={ball color=red}}
\newcommand{\epingle}[3]{
\coordinate[rotate=#2,yshift={#3*0.375cm}] (e) at #1;
\coordinate[shift={++(60:0.75)}] (g) at (e);
\begin{scope} [scale=1.5]
 \begin{scope}[rotate=-30]
   \coordinate[shift={++(30:0.75)}] (h) at (e);
   \draw[ombrePunaise,line cap=round,line width=4pt] (e) -- ++(60:0.75);
   \fill [ombrePunaise,rotate=-30,scale=0.5] (h) ellipse (.65 and .3) ;
   \fill [ombrePunaise,rotate=60,scale=0.5] (h) ++(0.4,0) ellipse (.4 and .3);
   \fill [ombrePunaise,rotate=60,scale=0.5] (h) ++(0.8,0) ellipse (.2 and .4);
 \end{scope}
 \draw[line cap=round,line width=4pt] (e) -- ++(60:0.75);
 \fill [punaise,rotate=-30,scale=0.5] (g) ellipse (.65cm and .3cm) ;
 \fill [punaise,rotate=60,scale=0.5] (g) ++(0.4,0) ellipse (.4 and .3);
 \fill [punaise,rotate=60,scale=0.5] (g) ++(0.8,0) ellipse (.2 and .4);
\end{scope}}
\tikzset{zlevel/.style={%
    execute at begin scope={\pgfonlayer{#1}},
        execute at end scope={\endpgfonlayer}
}}
\newlength{\mypostx}
\newcommand{\postit}[2]{%
    \setlength{\mypostx}{\marginparwidth}%
    \marginpar{%
    \begin{tikzpicture}[overlay,remember picture]%
            \begin{scope}%
            \node[opacity=0.7,inner sep=1em,rotate=#2,drop shadow,
                text width=\marginparwidth-2em](block)
                at (0.5\marginparwidth,0.){\color{black}#1};
            \fill[yellow!50!white] (block.north west) -- (block.north east)
                .. controls ($ (block.north east)+1.5*(block.west)$) ..
                  ($ 0.9*(block.south east) $)
                  .. controls ($ (block.south)$) ..
                  (block.south west) -- cycle;
            \node[inner sep=1em,rotate=#2,
                text width=\marginparwidth-2em]
                at (0.5\marginparwidth,0.){\color{black}#1};
            \epingle{($ (block.north)-(0.,1em) $)}{5}{0.5};
        \end{scope}%
    \end{tikzpicture}%
    }
}
